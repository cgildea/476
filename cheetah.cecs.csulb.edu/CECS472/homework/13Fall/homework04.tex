\input macros
\input cstuff
\headline{{\bf CECS 472\hfill Homework 4 \hfill Fall 2013}}
\footline{Nathan Pickrell \hfill 10 September 2013 (Week 3 Lecture 1) \hfill
Due: 12 September 2013 (Week 3, Lab 2)}
\parindent 0pt

Purpose: This is designed to familiarize you with information
retrieving system calls.

Build a program that takes 5 command line arguments
(that's counting the command).

If the number of arguments is not equal to 5, the program
should print a message and exit.
(Remember argument 0 is the name of the command.)

Argument 1: this should be the name of a machine.
Use this in a {\ltt{}gethostbyname} call.
If a {\ltt{}NULL} pointer is returned print a message;
otherwise print the IP address (use {\ltt{}inet_ntoa} as in lecture).

Argument 2: this should be the name of a service.
Use this in a {\ltt{}getservbyname} call.
If a {\ltt{}NULL} pointer is returned print a message;
otherwise print the port number of the service.
Assume the protocol is {\ltt{}"tcp"}.

Argument 3: this should be the port number of a service.
Use this in a {\ltt{}getservbyport} call.
If a {\ltt{}NULL} pointer is returned print a message;
otherwise print the name of the service.
Assume the protocol is {\ltt{}"tcp"}.

Argument 4: this should be a dotted IP number (such as
134.139.248.17. Call {\ltt{}inet_addr} to convert it into
a 4 byte internet number.
If you get {\ltt{}INADDR_NONE} back print a message (bad format);
otherwise; call {\ltt{}gethostbyaddr} with the 4 byte internet number
If {\ltt{}gethostbyaddr} returns a {\ltt{}NULL} pointer print a message;
otherwise print the name of the host.

{\ltt{}~volper/classes/472/shells/info.c} contains 
the code examples from the lecture
(you don't have to retype them).

Call your finished program {\ltt{}info.c}.
\bye
