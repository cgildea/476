\input macros
\input cstuff
\headline{{\bf CECS 472\hfill Homework 4 \hfill Fall 2004}}
\footline{Dennis Volper \hfill 13 September 2004 (Week 3 Lecture 1) \hfill
Due: 15 September 2004 (Week 3, Lab 2)}
\parindent 0pt

Purpose: This is designed to familiarize you with information
retrieving system calls.

Build a program that takes 4 command line arguments
(that's counting the command).

If the number of arguments is not equal to 4, the program
should print a message and exit.
(Remember argument 0 is the name of the command.)

Argument 1: this should be the name of a machine.
Use this in a {\ltt{}gethostbyname} call.
If a {\ltt{}null} pointer is returned print a message;
otherwise print the IP address (use {\ltt{}inet_ntoa} as in lecture).

Argument 2: this should be the name of a service.
Use this in a {\ltt{}getservbyname} call.
If a {\ltt{}null} pointer is returned print a message;
otherwise print the port number of the service.

Argument 3: this should be the port number of a service.
Use this in a {\ltt{}getservbyport} call.
If a {\ltt{}null} pointer is returned print a message;
otherwise print the name of the service.
(Assume the protocol is {\ltt{}"tcp"}.

Finally: call {\ltt{}gethostbyaddr} with the number
{\ltt{}0x868bf811}. You will have to do an {\ltt{}htonl}
on this and put it into an {\ltt{}int} so you can pass the
address to the routine.
If a {\ltt{}null} pointer is returned print a message;
otherwise print the name of the host.

{\ltt{}~volper/classes/472/shells/info.c} contains 
the code examples from the lecture
(you don't have to retype them).
\bye
