\input macros
\input cstuff
\headline{{\bf CECS 472\hfill Project 14 \hfill Fall 2004}}
\footline{Dennis Volper \hfill 25 October 2004 (Week 9 Lecture 1)\hfill 
Due: 27 October 2004 (Week 9 Lecture 2)}
\parindent 0pt

Purpose: This assignment uses the multiservice technique
topics covered in Chapter 15.
You will build a multiservice server
using the technique from Chapter 15.

Your server will provide 3 services.

The UDP {\ltt{}timed} service (see homework 7) will be provided
on your port 5001.

The TCP {\ltt{}Ttimed} service (see homework 8) will be provided
on your port 5002.

The TCP {\ltt{}browserd} service (see homework 9) will be provided
on port 5003.

To make these ports work, place them into the {\ltt{}servent} array
as the service names.

Recommendation: Start with Comer's code.
Cut and paste your service procedures in front of Comer's code.
Finally, Modify the service array entries.
If your service code isn't in procedures, you will need to put it 
into procedures.

{\bf Submit:} The a fully commented copy (print-out) of the source code for the 
server.
In addition,
the source code for the server must be placed in your home directory
in a file named {\ltt{}multid.c}.

{\bf Testing:}
Since we don't have individual port numbers this time; you will
collide with each other if you run your servers on the same machine.
For this reason the following rule applies:
Run your server {\it only} on the machine whose keyboard you are using.

Remember the starndard Comer switch in the clients allows them
take 3 parameters on the command line.
So you could run (if your server is on {\ltt{}lab76})

{\ltt{}timec lab76 5001}

Forcing the client to connect to port 5001 on the lab76. 



\bye
