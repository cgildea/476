\input macros
\input cstuff
\headline{{\bf CECS 472\hfill Homework 16 \hfill Fall 2013}}
\footline{Nathan Pickrell \hfill 29 October 2013 (Week 10 Lecture 1)\hfill 
Due: 31 October 2013 (Week 10 Lecture 2)}
\parindent 0pt

Purpose: This assignment requires an in-depth understanding of
using multicast messages over UDP.
You will build a time sender that sends multicasts the time
every  second
on your personal port using multicast address {\ltt{}224.0.1.2};
and a multicast time receiver
that receives times on your personal port using the same multicast
address.

Remember, the model here is different. 
The sender has the network
characteristics of a client (no bind), but it is sending
the time like a time server.
The receiver has the network characteristics of a server (bind
to a well known port), but it is receiving the time like
a time client.

The time will be sent using the {\ltt{}gettimeofday} format
used in your {\ltt{}timec.c} and {\ltt{}timed.c}.

Extended versions of the code from the lecture is located
in {\ltt{}~volper/classes/472/programs} in the files
{\ltt{}multisend.c} and {\ltt{}multirecv.c}.

{\it THE MULTICAST TIME SENDER}

The {\ltt{}mtimes} sender will send the time in {\ltt{}gettimeofday} format
every second to all receivers listening on your well known port. 
This means, use a {\ltt{}while(1)} loop with a {\ltt{}sleep(1)}
command at the end so it cycles once a second.

{\it THE MULTICAST TIME RECEIVER}

The {\ltt{}mtimer} receiver will listen for multicasts
on your well-known UDP port.
The server will enable the receipt of multicast messages.
Whenever it receives a message from a client it will
print the time (see {\ltt{}timec.c} for the format)
So you can better see what is going on
it must also print the address and port of the sender
(you have already done this in a previous assignment).

This means a loop with a {\ltt{}recvfrom}.
Unlike other programs, there is no {\ltt{}sendto}.

{\bf Submit:}
A print out of the sender and receiver.
In addition,
the source code for the sendder must be placed in your home directory
in a file named {\ltt{}mtimes.c} and
the source code for the receiver must be placed in your home directory
in a file named {\ltt{}mtimer.c}.

{\bf Discussion:}

When you send a multicast, each receiver on the subnet should
receiver the message.
Remember, UDP is ``unreliable" so occasionally one
of messages may not get delivered.

We are not allowing multicasts to go through gateways to avoid
complaints from the rest of the net; so, as in your broadcast homework,
your test will need to use machines on the same subnet.

To test this, start one sender and 3 receivers on the same subnet.
It doesn't really matter which you start first.
It's kind of interesting to start a receiver first, then the
sender, then a couple more receivers.

Hint: get it to work with one receiver first.
\bye
