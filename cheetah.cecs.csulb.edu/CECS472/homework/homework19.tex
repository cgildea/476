\input macros
\input cstuff
\headline{{\bf CECS 472\hfill Homework 19 \hfill Fall 2013}}
\footline{Nathan Pickrell \hfill 14 November 2013 (Week 12 Lecture 2)\hfill
Due: 19 November 2013 (Week 13 Lab 1)}
\parindent 0pt

Purpose: To gain an in-depth understanding of
remote procedure calls.

You will build a remote program and a client that makes remote procedure calls 
to that program.
Your remote procedure calls will use TCP.

Your remote program (server) will provide several remote procedures.
Your rpc client will allow the user to call any of these procedures.
For the remote program you must use your personal RPC program
number.
Compile and run the {\ltt{}Uniq_Id.c} program 
(from {\ltt{}~volper/classes/472/programs/rpc_example})
to get your program number.

Overview: You are writting an rpc (client/server) program for an automobile
repair shop. When a customer brings a car for repair, information about
that car is entered into a remote data base.
This data base represents the queue of cars awaiting repair. 
When a technician becomes available, a car from the queue is given to that 
technician for repair (and that car is removed from the queue).

{\it THE SINGLE MACHINE PROGRAM} 

In accordance with the RPC development procedure covered in class a
single process (non-network, non-rpc) version of the program must be
built first.
Use the {\ltt{}repair_client.c} and {\ltt{}repair_services.c} programs
found in the {\ltt{}programs/rpc} directory
You should copy that program, compile and run it so you know what it does.
The program is already separated into two files for your convenience.
When building the RPC version of the program you must
build the inteface procedures with the same interface
for the services as in {\ltt{}repair_services.c}

You must make each of the services provided by the program
available as remote procedures in a remote program.

The emphasis here is on the rpc implementation.
Keep the same (simplified) user interface as I have provided.

{\it THE RPC VERSION}

Your first job is to define the interface or {\ltt{}.x} file.
Make sure the number of values passed to each of the remote procedures
corresponds to the number of parameters the corresponding
sevice procedures requires.
Make sure the number of values returned by each of the remote procedures
corresponds to the number of values the corresponding
sevice procedures delivers back to the user.
Remember any service procedure parameter that is a pointer may be
used to either send or return values. You have to analyze the code
to determine this.

{\it THE RPC SERVER PROGRAM}

Since I give you the service procedures;
your job is to implement the the server-side interfaces to provide
the correct packing and unpacking necessary between the server procedures
and the rpc calls.

{\it THE RPC CLIENT PROGRAM}

The rpc client main program will take one optional command arguement,
the name of the host to which it is to make its rpc calls.
This means copying in the Comer switch with the service
option deleted into the client program.
You also must add the call to create the handle to the server.

Your other job is to implement the the client-side interfaces to provide the
correct packing and unpacking necessary between the client and the rpc calls.

{\bf Submit:} The fully commented copy (print-out) of the source code
for the following files:
{\ltt{}repair.x},
{\ltt{}repair_cif.c},
{\ltt{}repair_sif.c},
{\ltt{}repair_client.c},
In addition the source code for these files should be placed in your home
directory.
Note: {\ltt{}repair_client.c} is going to be mostly my code,
but I want to make sure you have made the modifications.
{\ltt{}repair_services.c} should not be changed
and should not be submitted.

Note that the files:
{\ltt{}repair_clnt.c},
{\ltt{}repair_svc.c},
{\ltt{}repair_xdr.c},
{\ltt{}repair.h},
will be automatically generated, but since you don't do anything to them
you should not submit them.

{\bf Reminder 1:}
Your parameters in 
{\ltt{}repair_client.c}
{\it must} use the types and modes specified for the single machine program.
Do not change the prototypes.

{\bf Reminder 2:}
Remember that {\ltt{}repair_sif.c}
will need prototypes for each of the services functions.
They should be the same as those in the {\ltt{}repair_client.c}.

\bye
