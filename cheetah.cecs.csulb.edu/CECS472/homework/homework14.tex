\input macros
\input cstuff
\headline{{\bf CECS 472\hfill Homework 14 \hfill Fall 2013}}
\footline{Nathan Pickrell \hfill 22 October 2013 (Week 9 Lecture 1)\hfill 
Due: 24 October 2013 (Week 9 Lecture 2)}
\parindent 0pt

Purpose: This assignment uses the multiservice technique
topics covered in Chapter 15.
You will build a multiservice server
using the technique from Chapter 15.

Your server will provide 3 services.

The UDP {\ltt{}timed} service (see homework 7) will be provided
on your port 5001.

The TCP {\ltt{}Ttimed} service (see homework 8) will be provided
on your port 5002.

The TCP {\ltt{}browserd} service (see homework 9) will be provided
on port 5003.

To make these ports work, place them into the {\ltt{}servent} array
as the service names.

Recommendation: Start with Comer's {\ltt{}superd.c} code.
Cut and paste your service procedures into Comer's code.
If your service code isn't in procedures, you will need to put it 
into procedures.
If you paste your procedures after Comer's code; be sure you
add prototypes for your procedures before the service array declaration.
Finally, Modify the service array entries.

{\bf Submit:} The a fully commented copy (print-out) of the source code for the 
server.
In addition,
the source code for the server must be placed in your home directory
in a file named {\ltt{}multid.c}.

{\bf Testing:}
Since we don't have individual port numbers this time; you will
collide with each other if you run your servers on the same machine.
For this reason the following rule applies:
Run your server {\it only} on the machine whose keyboard you are using;
DO NOT run your server on {\ltt{}cheetah}.

Remember the standard Comer switch in the clients allows them
take 3 parameters on the command line.
So (if your server is on {\ltt{}lab76}) you could run your existing client as:

{\ltt{}timec lab76 5001}

to force the client to connect to port 5001 on the lab76. 

At a  minimum you should do the following sequence of tests
without restarting your multid: broswerc (do l, followed by q), Ttimec,
timec, Ttimec.
Make sure that a test doesn't cause the server to freeze up for later tests.

\bye
