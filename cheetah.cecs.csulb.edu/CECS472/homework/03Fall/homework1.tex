\input macros
\input cstuff
\headline{{\bf CECS 472\hfill Homework 1 \hfill Fall 2003}}
\footline{Dennis Volper \hfill 3 September 2003 (Week 1 Lecture 2)\hfill 
Due: 10 September 2003 (Week 2, Lab 2)}
\parindent 0pt

Purpose: This assignment is to familiarize you with the Unix systems at
CSULB, and some of the networking applications available to you.

This assignment will force you to read the on-line Unix manual.
For those of you who have never used Unix, this assignment may take a
considerable amount of time.
You will also need to know/learn some important non-network commands such as 
an editor {\ltt{}vi} and how to list files {\ltt{}ls}.

Submission: You will submit a written document.
This document and all submissions in this course will have a header
containing the following information:

Your name, your account number, if the document is a computer program
the header will contain the name of the file containing that program.

To complete this assignment you will need to read the manual entries
for several programs.

1) rlogin/telnet: ({\it Unix manual entries:} {\ltt{}telnet}, {\ltt{}rlogin})

There is an account setup on panther.net.cecs.csulb.edu.
It contains a secret message, 
that is it contains a file named {\ltt{}message} that is located
in a directory called {\ltt{}hidden}. This message is
unreadable unless you rlogin or telnet to that account.
The account name is {\ltt{}temp472}, the account password is {\ltt{}pass472}.
Use both {\ltt{}rlogin} and {\ltt{}telnet} to get to account.
When you attempt to rlogin, if you are asked for the password twice, it means
that you have used the rlogin command incorrectly (re-read the manual entry).

Submit: 
a) what commands did you use (exactly what was the {\ltt{}rlogin} command
and the {\ltt{}telnet} command that you typed),
b) the secret message.

2) ftp: 
({\it Unix manual entries:} {\ltt{}ftp})

There are numerous files available on {\ltt{}panther.net.cecs.csulb.edu}.

Use ftp (from any machine other than {\ltt{}panther} to get the file 
pub/README.
Hint: use the login ``anonymous".

Submit: 
a) a copy of that file.
b) the command sequence you used to get that file.

3) rcp: 
({\it Unix manual entries:} {\ltt{}rcp}, {\ltt{}rlogin})

Login to {\ltt{}cheetah} and attempt to rcp the secret message from the 
{\ltt{}temp472} account into your account (i.e., from {\ltt{}panther}
to {\ltt{}cheetah}). 
This attempt fails.
Fix the 472 account so it will allow you to rcp the file.
(The same fix that lets you avoid specifying the password also enables
the rcp.)
When you get it to work, leave your fix there so the instructor can admire it.

Submit:
a) the error message on the failed attempt.
b) a one line description of what you did to fix the 472 account so the rcp 
worked.
c) the exact rcp command you used that worked to get a copy of the message.

4) finger:
({\it Unix manual entries:} {\ltt{}finger})

Remotely (from any machine in the lab other than {\ltt{}panther})
examine the {\ltt{}temp472}.
Submit: a )what is the official name of the 
owner of that account? b) what is reported for the roome number?
c) the exact command you ran to get this information,
d) the machine on which you ran that command.

Make a piece of information about you available to finger,
such as ``I will graduate in Spring 1995". 
Due to security restrictions you will only be able to view this information
from certain machines.

Submit:
a) the name of the home directory.
b) a one line description of {\it how} you made the information available.
(Don't tell me {\it what} the information was, 
I'll find that by running finger.)

5) Log into cheetah and run the register program (so I know who is using
which acocunt). (No submission on this one.)

\bye
