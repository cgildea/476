\input macros
\input cstuff
\headline{{\bf CECS 472\hfill Homework 7 \hfill Fall 2013}}
\footline{Nathan Pickrell \hfill  19 September 2013 (Week 4 Lecture 2) \hfill
Due: 24 September 2013 (Week 5, Lab 1)}
\parindent 0pt

Purpose: This assignment is to gain an in-depth understanding of
a UDP client-server program.

You will build a connectionless time client and an iterative connectionless
time server using the techniques found in Chapter 9 of the text.
(You will use UDP.)

Submission: You must submit a print out of both the client and the server.
In addition you must place the source code for both the client and
the server in your home directory in files called {\ltt{}timec.c}
and {\ltt{}timed.c}, respectively. I will access these files to check
the correctness of your programs.

Port numbers: You must use your personal well-known-port number.
Copy the file\hfill\break {\ltt{}~volper/classes/472/shells/get_port.c}
into your home directory. The {\ltt{}get_port} procedure in this file 
returns your personal port number.
Use an include directive to make the procedure available,
and call this procedure in both your client and server at the point
the {\it default} service is set up (see the lecture notes).

\noindent
{\it THE TIME SERVER}

Your time server will listen on your well known UDP port.
Whenever it receives a message from a client it will call {\ltt{}gettimeofday}
and return the {\ltt{}tv_sec} and {\ltt{}tv_usec} values to the client.
The call is similar to the {\ltt{}time} in Comer's example,
except it returns a structure.
It does have a second parameter which is unused, pass a {\ltt{}NULL}
for that parameter.
Don't forget to use the ampersand on the first parameter
(see Comer's example).

You are returning 8 bytes; you will need an 8 byte array.

Sequence:

1) Get a request.

2) Use the {\ltt{}gettimeofday} call.

3) Print (to the screen) the values of
{\ltt{}tv_sec} and {\ltt{}tv_usec} in hexidecimal.

4) {\ltt{}htonl} the {\ltt{}tv_sec} and {\ltt{}tv_usec},
go ahead and put them right back into the structure.

5) Use two calls to {\ltt{}memcpy}, to copy the sec into
the array starting at position 0 and
the usec into the array starting at position 4.

6) Send the array (8 bytes).

\noindent
{\it THE TIME CLIENT}

Your time client will contact your server and get the time on the server's 
machine.
The client will send a  message and receive as a reply the
time on the server's machine given as the two {\ltt{}gettimeofday} numbers
({\ltt{}tv_sec} and {\ltt{}tv_usec}).

You are receiving 8 bytes; you will need an 8 byte array.

You are receiving a time of day you are required to
use a {\ltt{}timeval} structure to contain the final answer.

Sequence:

1) Send a request to the server.

2) Read the reply.

3) Use two calls to {\ltt{}memcpy} to copy the sec and usec into
the ({\ltt{}tv_sec} and {\ltt{}tv_usec}) fields of a {\ltt{}timeval}
structure.

4) {\ltt{}ntohl} the {\ltt{}tv_sec} and {\ltt{}tv_usec},
go ahead and put them right back into the structure.

5) Print (to the screen) the values of
{\ltt{}tv_sec} and {\ltt{}tv_usec} in hexidecimal.

\bigskip
Recommendations: (not requirements)

Server)
A good starting point is Comer's {\ltt{}UDPtimed.c}.
Don't forget to change your default {\ltt{}service} to use
{\ltt{}get_port}.
Get rid of the Unix Epoch stuff, have both your client and your server
use the {\ltt{}gettimeofday} time format.

Client) 
Start with your UDP client from your homework, it's most of the solution.
\bye
