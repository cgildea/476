\input macros
\input cstuff
\headline{{\bf CECS 472\hfill Project 3 \hfill Spring 2004}}
\footline{Dennis Volper \hfill  9 February 2004 (Week 3 Lecture 1) \hfill
Due: 23 February 2004 (Week 5, Lab 1)}
\parindent 0pt

Purpose: This assignment is to gain an in-depth understanding of
a client-server program.

You will build a time client and an iterative connectionless time server
using the techniques found in Chapter 9 of the text.
(You will use UDP.)

Submission: You must submit a print out of both the client and the server.
In addition you must place the source code for both the client and
the server in your home directory in files called {\ltt{}t472c.c}
and {\ltt{}t472d.c}, respectively. I will access these files to check
the correctness of your programs.

Port numbers: You must use your personal well-known-port number.
Copy the file\hfill\break {\ltt{}~volper/classes/472/shells/get_port.c}
into your home directory. The {\ltt{}get_port} procedure in this file 
returns your personal port number.
Use an include directive to make the procedure available,
and call this procedure in both your client and server at the point
the {\it default} service is set up.

\noindent
{\it THE TIME SERVER}

Your time server will listen on your well known UDP port.
Whenever it receives a message from a client it will call {\ltt{}gettimeofday}
and return the {\ltt{}tv_sec} and {\ltt{}tv_usec} values to the client.

To give a little visibility to what the server it doing;
after it returns the time, it will use the {\ltt{}inet_ntoa} call to
print (to the screen) the internet number of the client to which the time
was returned.
The internet number is in the return address,
the {\ltt{}inet_ntoa} call was used in an example in the Chapter 6 lecture.
It wasn't used with the return address in that example, so you will
have to put things together a little differently.
(Hint: it can be done entirely within a {\ltt{}printf} statement.)

\noindent
{\it THE TIME CLIENT}

Your time client will contact your server and get the time on the server's 
machine.
The client will send a  message and receive as a reply the
time on the server's machine given as the two {\ltt{}gettimeofday} numbers
({\ltt{}tv_sec} and {\ltt{}tv_usec}).
The client will report two items:

1) The {\it rtt}, that is the round trip time between the machines
in {\bf microseconds}. (The same thing you did in homework 2.)

2) The {\it offset}, that is the difference,
between the client machine's clock and server machine's clock in 
{\bf microseconds}.
Thus, positive offsets will mean that the client's clock is faster than the 
server clock and negative offsets mean that it is slower.
You client shall use the following formula for the offset:
\hfill\break
{\ltt{}(client_start_time - server_time) + (rtt/2)}.
Subtraction the times using the same technique as you did for computing
the rtt in project 2.

You only need one round trip and offset (not three as in homework 2).

Requirements:

You have 8 bytes to ship from the server to the client.
After converting to network order, you will use two 
{\ltt{}memcpy} calls to put these
into a single 8 character array and then ship the eight bytes in one write.
At the client end, do one read, use two {\ltt{}memcpy} calls to unpack,
then convert to host order.

\bigskip
Recommendations: (not requirements)

Server)
A good starting point is Comer's {\ltt{}UDPtimed.c}.
Don't forget to change your default {\ltt{}service} to use
{\ltt{}get_port}.
Get rid of the Unix Epoch stuff, have both your client and your server
use the {\ltt{}gettimeofday} time format.

Client) 
Start with your UDP client from homework 2, it's most of the solution.
Remove the {\ltt{}for} loop.
Switch the default service to use {\ltt{}get_port}.
Switch the code to read and unpack the time values.
Add the offset calculation.


Testing:
If you want, you can use {\ltt{}ctime} to print out the seconds part of
the time you get from the server to make sure that what you unpacked
to make sure it is reasonable. Don't forget to do at least one test
with your client and server running on different machines.

Note:
This is UDP; you can lose packets.
As in homework 2, a lost packet will cause your client to hang.
Don't worry if your client hangs occasionally.

%Minimum Testing:
%\hfill\break
%client: {\ltt{}cheetah} \qquad server: {\ltt{}cheetah}
%\hfill\break
%client: {\ltt{}cheetah} \qquad server: {\ltt{}puma}
%\hfill\break
%client: {\ltt{}puma} \qquad server: {\ltt{}jaguar}
%\hfill\break
%client: {\ltt{}labxx} \qquad server: {\ltt{}jaguar}

\bye
