\input macros
\input cstuff
\headline{{\bf CECS 472\hfill Homework 2 \hfill Spring 2004}}
\footline{Dennis Volper \hfill 2 February 2004 (Week 2 Lecture 1) \hfill
Due: 11 February 2004 (Week 3, Lab 2)}
% Week 2, Lab 1 issue except for Labor day
\parindent 0pt

Purpose: This assignment is to introduce you to the different types of clients
used in network programming and to programming with Unix sockets.

This assignment will help you understand the examples and discussion
in Chapter 7 of the text. 
Task 1 is designed to improve your understanding UDP.
Task 2 will improve your understanding of TCP.
It will also require you to read the manual entry
for {\ltt{}gettimeofday}.

Submission: You will submit two programs and one ``listing of results"
(see Testing).
This document and all submissions in this course will have a header
containing the following information:

For each file (program)submitted as a solution to a homework assignment you 
must include your name, your account number, and the name of that file in
a comment at the start of the program.
The files must be given the names specified, and placed
{\it directly in the home directory of your class account}
(not in a subdirectory).
Use copies of my versions of Comer's code.
Any include commands for Comer files should be from the local directory:
For this program, you are to explicitly place the close calls in
the correct place in your program, 
even in those cases that the exit command would perform them.

{\ltt{}#include "connectUDP.c"  /* copied from comer_examples */}

Tasks:

1) Modify the code for {\ltt{}UDPtime.c} on pages 86--87 of Comer and Stevens
so that it also computes the round trip time.
Change the code so that it computes the amount of time between when it sends
until it receives a reply.
Because the time is fast you will use the microseconds capability of 
{\ltt{}gettimeofday}.
You will a real structures for the {\ltt{}timeval}
However, since you are measuring round trip time, the time zone is
immaterial and a {\ltt{}NULL} can be used.

Your program will make three requests and report the three times
(for loop).
You technically are not modifying any of the networking code, 
you are working around it,
but you must understand the networking code to correctly place the
loop, time calls, connect and close.
Turn in the program and report the results of the three tests asked for 
below (i.e., nine separate times).
The file should be called {\ltt{}myUDPtime.c}.

2) Modify your code from part 1 so that it uses TCP instead of UDP to access 
time.
Be careful where you put the before and after time calls, an understanding
of how TCPtimed works is important in determining their correct location.
Turn in the program and report the results of the three tests asked for 
below (i.e., nine separate times).
Again, be sure you correctly place the
loop, time calls, connect and close.
(And be sure you don't have unnecessary calls.)
The file should be called {\ltt{}myTCPtime.c}.

Notes on Tasks:

In {\ltt{}~volper/classes/472/comer_examples} you can find the modified code 
for the examples in the book. You should use copies of this code.
Don't modify this code, don't change the names of the include files,
don't use any include files of your own,
and be sure your solutions have the exact file names required.
This is so that I can automatically grab, recompile and test your programs
with a minimum of problems.

Compiling:

On {\ltt{}cheetah}/{\ltt{}panther}/{\ltt{}labxx}: {\ltt{}gcc myUDPtime.c}
\hfill\break
On {\ltt{}puma}: {\ltt{}gcc myUDPtime.c -lsocket -lnsl}
\hfill\break
(Because the socket stuff is not in the standard library on the Suns.)

Testing:
You are required to time the following connections and report your results:

{\ltt{}labxx} to {\ltt{}labxx}

{\ltt{}labxx} to {\ltt{}puma}

{\ltt{}labxx} to {\ltt{}panther}

That is, 18 times, 3 UDP and 3 TCP for each of the above tests.
You should submit the 18 numbers with your program listings,
they may be hand-written onto the program listings.
\bye

% move to homework3
\vfill\eject
Notes on further testing:

The above tests do not indicate your code is completely correct.
The following tests are to help make sure that it is before I grade it.
These tests should not be submitted, they are to help keep you from losing
points for an incorrect program.
\hfill\break
First, connect from a computer to itself (if it doesn't work in this case
you don't have a chance of working between computers).
\hfill\break
Second, connect between two computers that have their internal integers 
formated in the same order as network standard (from {\ltt{}labxx} to
{\ltt{}cheetah}).
\hfill\break
Third, connect between two computers that have their internal integers 
formated in the opposite order as network standard ({\ltt{}puma}).
\hfill\break
Fourth, connect (test both directions) between two computers that have their 
internal integers formated in the different order (e.g., {\ltt{}cheetah} and
{\ltt{}puma}).
\hfill\break
Fourth, connect to computers that are ``distant", that is, through
several gateways.
{\ltt{}aardvark} is several gateways away although it is not
a good test for large round trip times because it is still in COE.


