\input macros
\input cstuff
\headline{{\bf CECS 472\hfill Project 4 \hfill Spring 2004}}
\footline{Dennis Volper \hfill 18 February 2004 (Week 4 Lecture 2)\hfill 
Due: 1 March 2004 (Week 6 Lecture 1)}
\parindent 0pt

Purpose: To gain an in-depth understanding of
a concurrent connection-oriented server (Chapter 11).

You will build a client and a concurrent connection-oriented server.
Your client-server combination will provide the ability browse
the file system on the server's machine.
You will provide 3 capabilities 
(1) {\bf c} to change directories and
%(2) {\bf p} to print the current working directory and
(2) {\bf g} to get/retrieve a file
(3) {\bf l} to list the entries in the current directory.
You will use a connection-oriented multiple process server (TCP).

{\it THE BROWSERD SERVER}

The {\ltt{}browserd} server will listen on your personal well-known TCP port.
The server will accept as many clients as want to connect.
Each time it accepts a connection from a client,
the server will fork a process to handle that client. 
When the client closes the socket, the (slave) server should exit.

The server will handle three types of requests from the client:
change directory,
%report the current directory,
get (retrieve) a file,
and list the contents of the current directory.

When the client requests a directory change,
the client must send to the server the command and a null terminated
string that contains the command and the name of the directory. 
(This is a request for a directory change by the server on the server's 
machine.)
The server will attempt to change to that directory (it might not be able
to due to a bad directory name or lack of permission).
%The server should use the {\ltt{}chdir} system call to attempt this change,
%passing it the name portion of the string it received from the client.
The server will not send any reply to the client,
hence the client will not be told if the directory change worked.

When the client requests a listing of the server's current directory,
the client sends only the command.
The server, gets a listing of the directory and sends this listing
back to the client.
The end of the listing shall be indicated by the sending of a string
containing a single null.
Nulls must not be sent during (in the middle of) the listing. 

%When the client requests a report of the current directory,
%the client sends only the command a single letter followed by a null.
%The server uses the {\ltt{}getcwd} command to get the name of the current
%directory.
%The server sends that directory name to the client as a null terminated string.

When the clients requests a get of a file,
the client must send to the server the command and a null terminated
string that contains the command and the name of the file to be
retrieved. 
The server will attempt to open that file (it might not be able
to due to a bad name or lack of permission).
The server will use the {\ltt{}stat} command to see how large the
file is.
It will ship the file size (4 bytes, network order) to the client;
followed by the file.

Note: the list and get commands use the two different models
of transferring information over the network,
{\it list} uses a terminator,
{\it get} sends a byte count.

{\it THE BROWSER CLIENT}

When started, the client will open a TCP connection to the server.
The user of the client will be given four options,
%{\ltt{}c}, {\ltt{}p}, {\ltt{}l}, and {\ltt{}q}.
{\ltt{}c}, {\ltt{}g}, {\ltt{}l}, and {\ltt{}q}.
You shall use the user interface provided in the shell found in
\hfill\break
{\ltt{}~volper/classes/472/shells/browser.c}.
%If the user types {\ltt{}c}, {\ltt{}p}, or {\ltt{}l}, the client will
If the user types {\ltt{}c}, {\ltt{}g}, or {\ltt{}l}, the client will
send the entire command to the server.
If the user typed {\ltt{}q}, the client shall exit
(you should close the socket before exiting, but that is all).
If the user typed {\ltt{}l}, the client shall
print the directory listing the server returns.
%If the user typed {\ltt{}p}, the client shall
%print the name of the directory the server returns.
If the user typed {\ltt{}g}, the client shall
save the file that is transferred from the server.

%In addition to the listings returned by the server
%markers shall be printed (see the discussion of
%boundaries below).

{\bf Submit:} The a fully commented copy (print-out) of the source code for the 
client and the server.
In addition,
the source code for the client must be placed in your home directory
in a file named {\ltt{}browser.c} and
the source code for the server must be placed in your home directory
in a file named {\ltt{}browserd.c}.

{\bf Directions and Discussion:} 
You are given a shell ({\ltt{}browser.c}). 
That shall has a (extemely primitive) user interface.
Your program shall have the same user interface.
You program must print the directory listing in exactly the same
format that the shell does.
The shell assumes the user types the command correctly,
(for example that c is followed by a space and a directory name)
you also should make this assumption
(this is a user interface problem and we are concentrating on networking).
Compile and run the shell to see how it behaves.
So what is missing from the shell;
the shell browses directories on the local machine.
You need a client/server pair that allows you to browse directories on
a remote machine (you used ftp to do this in assignment 1).
So you must turn the local program into network program.
Since part of the shell belongs on the client and part belongs
on the server, I recommend you make a copy of the file
(call it {\ltt{}browserd.c}) that will be modified to become the server.

To the client copy you will need to add the 
capability to connect to a server, 
including handling the command line (the case argv stuff from Comer)
as was done in your time client
(I recommend copying the code from your time client).
You need to send commands to the server,
%and in the case of {\ltt{}l} and {\ltt{}p}; read and print the answer.
and in the case of {\ltt{}l} read and print the answer;
and in the case of {\ltt{}g} read and save the answer into a file.
Do not send the {\ltt{}q} command; instead close the socket and exit.
When you print the directory, you can't get the whole directory before 
printing it (it can be arbitrarily large), 
you must print it as it arrives (like the echo client does).
Similary when you get (retrieve) a file you will need to save it to disk.
To reduce the risk of {\it get} clobbering a file, when the file is
saved it will always be given the name {\ltt{}xfer.tmp}; this means
that a second get will overwrite the first; but in the interest of
safety we will tolerate that.

In the server copy you need to add the passiveTCP and fork stuff.
A good way is to copy in the Comer code from chapter 11 and make the shell 
into the server procedure.
You read the command from the socket,
%run it, and ship the answer ({\ltt{}l} and {\ltt{}p}) back.
run it, and ship the answer ({\ltt{}l} and {\ltt{}g}) back.
If you get end of file, you close/exit.
In the case of the {\ltt{}l} command, the shell uses a {\ltt{}printf}
you server doesn't want to print it, it wants to ship it.
You should use {\ltt{}sprintf} to have the {\ltt{}printf} go to a string
({\ltt{}get_port} does this).

To both files you need to add the call to {\ltt{}get_port} so that
your client and server use your special well known port.

You must be careful with the communication between the client and the server.
Remember, {\it boundaries are not preserved} in TCP.
On the {\ltt{}c} command, make sure you have read the whole directory name
before attempting the {\ltt{}chdir}.
On the {\ltt{}l} command, the listing of the directory could be arbitrarily
large; the server must send the listing back to the client a line at a time.
The client will end up doing multiple reads. 
Even though the server writes a line at a time, several lines or portions
of a line may arrive in a read.
Again, {\it boundaries are not preserved}.
Since the null indicates the end of a directory listing,
you must be careful when you send the listing 
that you don't ship any nulls until the very end.
Programs that use read 1 byte at a time will be given a substantial penality.

%{\it Printing Markers:}
%To ensure you get the idea about no boundaries you are
%required to print markers in the middle of your listings.
%These markers will ``mess up" your listing, that is,
%they will place some extra ``stuff" into the output of
%your {\ltt{}l} and {\ltt{}p} commands.
%Requirement: each time the client does a network read,
%a {\it marker} will be printed on to the client's screen.
%The marker will consist of a vertical bar, a number
%and a second vertical bar ({\ltt{}printf("|%d|",n)}).
%The number will be the number of bytes read.

Be sure to test your server on more than one machine, and
for each server test, run clients on several different machines.
Be sure your tests exercise the concurrent nature of your server.
If you find your mistakes before submitting it, there is no penalty.
If it fails when I test it, there will be substantial penalty for
submitting a program that does not work.
Working in a single case (such as cheetah to cheetah) is not sufficient
for a client/server program.

When you test your get, try it on a text file which will allow you
to open it with the editor to ensure the contents got copied
correctly. (The method of transfer we are using doesn't distinguish
text from binary, so if you transfer text files correctly you will
transfer binary files correctly.

Notes: use English identifiers and add comments.
Bad style will be tolerated for the parts you copy from Comer,
but not for stuff you add.
Test your client and server on several machines.
Switch to a longer directory ({\ltt{}/etc}) and try it.
(You get points off if I find a problem,
but none if you find and fix it before turning it in!)
As with the previous programs you will discover there is not a lot to type.
You are given most of the pieces,
you have to put them together.
Allow your self enough time to debug,
little errors in a network program take a while to find.

Remember on {\ltt{}puma} to
add the network libraries by compiling with the command:

{\ltt{}gcc browser.c -lsocket -lnsl -o mybrowser}
\bye
