\input macros
\input cstuff
\headline{{\bf CECS 472\hfill Homework 6 \hfill Spring 2004}}
\footline{Dennis Volper \hfill 3 March 2004 (Week 6 Lecture 2)\hfill 
Due: 15 March 2004 (Week 8 Lecture 1)}
\parindent 0pt

Purpose: This assignment requires an in-depth understanding of
using broadcast messages over UDP.
You will build a time client that sends broadcasts and a time
server that listens for broadcasts.

{\it THE BROADCAST TIME SERVER}

The {\ltt{}btime} server will listen on your well-known UDP port.
The server will enable the receipt of broadcast messages.
Whenever it receives a message from a client it will send
the time back to that client (gettimeofday).

Except for responding to broadcasts, this is the same as
your homework 3 server.

{\it THE BROADCAST TIME CLIENT}

The {\ltt{}btime} client will accept input from either the
network or the keyboard.
You will need to put in a {\ltt{}while(1)} loop and
a select statement.
Except for using UDP broadcasts,
this part is similar to your homework 5 client.

If there is input from the keyboard, only the first
letter of the input is examined.
If that letter is a `q', the client exits.
If that letter is a `b', the client sends a broadcast
message to all servers listening on your well-known port.
All other keyboard inputs are ignored.

If there is input from the network, it is 8 bytes of time
coming in from some server that received your broadcast.
Unpack (like you did in homework 3) and print ({\ltt{}%x %x}) the
time the server sent to you (seconds, milliseconds).
(I like hexadecimal here because you can more easily see any
byte swap problems.)
Also print the address ({\ltt{}inet_ntoa}) of the server
you got the response from (like you did in homework 3).
Round trip time and offset do not exist in this project.

{\bf Submit:} The a fully commented copy (print-out) of the source code for the 
client and the server.
In addition,
the source code for the client must be placed in your home directory
in a file named {\ltt{}btimec.c} and
the source code for the server must be placed in your home directory
in a file named {\ltt{}btimed.c}.

{\bf Discussion:}

When you send a broadcast, each server on the subnet should
send a response.
For example, if you start three servers and your
client sends a broadcast (user types `b'),
then the client should print 3 times, one
from each of the servers.
Remember, UDP is ``unreliable" so occasionally one
of the servers may not respond.

To test this, start 3 servers and one client on the same subnet.

There are two subnets in the lab, both with netmask 255.255.255.224.
(That's 32 host number per subnet.)
These subnets are 134.139.248.64 and 134.139.248.32.
Make sure all three of your clients and your server are on the same subnet.
You cannot use {\ltt{}cheetah} for testing this assignment, it is not
on one of the above subnets.
\bye
