\input macros
\input cstuff
\headline{{\bf CECS 472\hfill Homework 2 \hfill Fall 2010}}
\footline{Dennis Volper \hfill 1 September 2010 (Week 1 Lecture 2) \hfill
Due: 8 September 2010 (Week 2, Lab 2)}
% Assigment is issued the second session of the course
\parindent 0pt

Purpose: This assignment reviews the creation of processes using
the fork system call.

The first fork example shown in class can be used as a starting point.
You may get it from 
\hfill\break
{\ltt{}~volper/classes/472/shells/fork.c}

Modify the fork program so that:

1) if it is the child it prints ``C" in front of the number
if it is the parent it prints ``P" in front of the number.
(Note: you will have to use the if statement from the second fork
example.)

2) The loop only goes to 10.

3) If it is the child, just after it prints the number, it sleeps for 2 seconds,
if it is the parent , just after it prints the number, it sleeps for 1 second.
(See {\ltt{}man 3 sleep}.)

Call your finished program {\ltt{}fork.c}.

\bye
