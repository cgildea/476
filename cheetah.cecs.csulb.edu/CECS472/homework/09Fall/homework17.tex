\input macros
\input cstuff
\headline{{\bf CECS 472\hfill Homework 17 \hfill Fall 2009}}
\footline{Dennis Volper \hfill 4 November 2009 (Week 10 Lecture 2)\hfill 
Due: 9 November 2009 (Week 11 Lecture 1)}
\parindent 0pt

Purpose: This assignment is designed to familiarize you
with the simplest form of preallocation.

{\it THE PREALLOCATED BROWSER SERVER}

You will build a browser server that uses preallocation.
You server will preallocate 3 (total) servers,
each of which is willing to do an accept.
Three means one original plus two you forked.

Start with a copy of the {\ltt{}browserd.c} you submitted.
Call your new verion {\ltt{}pbrowserd.c}.

{\it THE BROWSER CLIENT}

Use the {\ltt{}browserc.c} you submitted.
Do not change it.

{\bf Testing:}
Try to start 4 clients.
The first three should start fine;
the fourth should hang until one of the first three exits.

{\bf Submit:}
A print out of the server.
In addition,
the source code for the server must be placed in your home directory
in a file named {\ltt{}pbrowserd.c}.

{\bf Discussion:}

You will need to rearrange some of the code.
In particular, the forks and associated switch
should be done during the startup code;
not inside the while loop of the main program.
The while loop of the main program needs to 
become that for an iterative server;
an unconditional accept followed by a call to
the service procedure.
The service procedure does not need to be changed.

I selected 3 to because it is small enough to be tested.
A better server would preallocated and terminate servers much like a web server,
but that is too much code for a short assignment.

The behavior of this server reveals some items about the {\ltt{}connect} call.
In particular, the {\ltt{}connect} is only losely ties to the {\ltt{}accept}
in that the {\ltt{}connect} of the client will complete before the server
does an {\ltt{}accept};
as long as:
(1) the server has done a {\ltt{}listen} and (2) the queue length ({\ltt{}QLEN}) hasn't
been exceeded.
In effect, the client anticipates that an {\ltt{}accept} will be done.
Hence, the fourth client will ``hang" in the sense that the commands ({\ltt{}l}, {\ltt{}g}, {\ltt{}c})
will not work, but it will get to the prompt after the connect.
However, with a queue length of 5, the ninth client will see {\ltt{}connect} return a error. 
\bye
