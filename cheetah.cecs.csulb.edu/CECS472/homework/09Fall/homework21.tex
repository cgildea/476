\input macros
\input cstuff
\headline{{\bf CECS 472\hfill Homework 21 \hfill Fall 2009}}
\footline{Dennis Volper \hfill 30 November 2009 (Week 14 Lecture 1)\hfill
Due: 2 December 2009 (Week 14 Lecture 2)}
\parindent 0pt

Purpose: To use some of the advanced socket I/O capabilities.

You will build an upgraded TCP time client.
You will call your client {\ltt{}ATCPtimec.c}.
It will be an upgrade of the {\ltt{}TCPtime.c} client you built
for your homework.

As in you homework you will be using my server so,
the default service should {\ltt{}"5001"},
and the default host should be {\ltt{}"panther"}.

Remember, the {\ltt{}panther} server is perverse and it will try
to break up the reads and it will occasionally drop the connection.

Client Upgrades:

1) After setting up the socket, but before the read/reassemble loop,
the client should print the size of the socket's
send and receive buffers.

2) Also before the read/reassemble loop; set up the socket to be asynchronous
(non-blocking).

3) Inside the read/reassemble loop; if the read
gets an error of {\ltt{}EAGAIN} you should 
(a) print an {\ltt{}X}
(b) sleep for a second
(c) retry the read

Note: if the result of the read is {\ltt{}==0} (before you get 4 bytes)
or the error is not {\ltt{}EAGAIN}; the server has lost or dropped the
connection and you should exit without printing the time.

Expected client behavior:
When the server pauses in the middle of sending you the number
you should see an {\ltt{}X} or two come out.
The time you print should still be valid.

Submit: a printout of your {\ltt{}ATCPtimec.c} file.

A copy of this file must be located in your class home directory;
and have the name above.

\bye
