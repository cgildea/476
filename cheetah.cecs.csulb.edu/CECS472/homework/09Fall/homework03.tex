\input macros
\input cstuff
\headline{{\bf CECS 472\hfill Homework 3 \hfill Fall 2009}}
\footline{Dennis Volper \hfill 9 September 2009 (Week 2 Lecture 2) \hfill
Due: 14 September 2009 (Week 3, Lab 1)}
% Lecture 3
\parindent 0pt

Purpose: This is designed to familiarize you with the primitive client
and server. It basically forces you to read it through carefully.

Running: For your server, use the machine you are on;
for your client use the machine one higher.
For example, if you are on the keyboard of lab35, run server on
lab35 and your client on lab36.
The easy way to do this is to open a separate virtual session
(alt-F2) and telnet to the client machine.
Do not use the keyboard on lab51 or lab79 (because there are no
machines lab52 and lab80).

Discussion: This requirement will mean that each of you is running
their server on a different machine, so there will be no conflict
between your servers.
To do this you will need to to a little modification of the
hexidecimal constant of the client given in the lecture.
It is 0x868bf811; the last two digits need to be changed.
This constant applies if your are on the machine 134.139.248.17;
the last two hex digits (11) are equal to 17 in decimal.
Hence if you are on 134.139.248.99, you need the last two hex digits
to be 63 (the hexidecimal equivalent to 99).
You can tell internet number of the machine you are on by looking at the label
on the front of the machine.
If you have trouble with the decimal to hexadecimal conversion, let me
know, I'll assist you.

This project consists of adding several ``upgrades" to the client and
server and observing some output.

Client upgrades:
\hfill\break
1) What socket do you get? (printf the descriptor number)
\hfill\break
2) What value is returned by the connect, read and write calls (printf).
\hfill\break
3) Send a second message to the server, {\ltt{}"Message 2"}.
\hfill\break
4) Receive a second reply.
\hfill\break
5) Print the replies you receive from the server (printf).

Server upgrades:
\hfill\break
1) What sockets (master/client) do you get?
\hfill\break
2) What value is returned by the bind, listen, read and write calls.
\hfill\break
3) Receive a second message from the client.
\hfill\break
4) Send a second reply {\ltt{}"Reply 2"}.
\hfill\break
5) Print the messages you receive from the client (printf).

{\ltt{}~volper/classes/472/shells} contains copies
of the primitive client and server from lecture
(you don't have to retype them).

Call your finished client {\ltt{}primitive_client.c} and 
your finished server {\ltt{}primitive_server.c}.
\bye
