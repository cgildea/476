\input macros
\input cstuff
\headline{{\bf CECS 472\hfill Project 6 \hfill Spring 2002}}
\footline{Dennis Volper \hfill 18 March 2002 (Week 8 Lecture 1) \hfill Due: 10 April 2002 (Week 10 Lecture 2)}
\parindent 0pt

Purpose: To gain an in-depth understanding of
remote procedure calls.

You will build a remote program and a client that makes remote procedure calls 
to that program.
Your remote procedure calls will use TCP.

Your remote program (server) will provide three remote procedures.
Your rpc client will allow the user to call any of these procedures.
For the remote program you must use your personal RPC program
number.
Compile and run the {\ltt{}Uniq_Id.c} program 
(from {\ltt{}~volper/classes/472/programs/rpc_example})
to get your program number.

\bigskip
\noindent
{\it THE RPC SERVER PROGRAM}

The server program will be called {\ltt{}atmd} (automated teller machine).
The RPC server will provide three services called 
{\ltt{}deposit}, {\ltt{}withdraw} and {\ltt{}summarize}.

The atm has an array of two accounts (account 0 and account 1).
Initially {\ltt{}atmd} starts with all balances equal to 0.
All numbers are integers (no reals).

{\ltt{}deposit} must be called with two parameter, the first parameter
indicates how much the rpc client wants to deposit and
the second parameter indicates which account.
When called, {\ltt{}deposit} increases the balance of the appropriate
account by the amount the rpc client has deposited.

{\ltt{}withdraw} must be called with two parameters, the first parameter
indicates how much the rpc client wishes to withdraw and
the second parameter indicates which account.
When called, {\ltt{}withdraw} decreases the appropriate balance count by the 
amount the rpc client wants to withdraw, but the balance never is reduced 
below 0.
If the rpc client asks for more money than {\ltt{}atmd} currently
has, {\ltt{}atmd} delivers all the remaining balance (balance goes to 0).

{\ltt{}withdraw} returns an integer indicating the amount that the
rpc client was allowed to withdraw.
This integer is either equal to the amount the client asked for, or the
balance that {\ltt{}atmd} had left (if the client asked for too much).

{\ltt{}summarize} has no parameters.
When called, {\ltt{}summarize} returns the values in the balances array of
the {\ltt{}atmd}.

All integers are {\ltt{}long} and all rpc integer parameters and return
values will be {\ltt{}long}.
(This is to force you to xdr, you can't use strings!).

\bigskip
\noindent
{\it THE RPC CLIENT}

The rpc client will be called {\ltt{}atmc}.
It will take one parameter, the name of the host to which it is to make
its rpc calls.
Example: {\ltt{}atmc cheetah}

{\ltt{}atmc} provides the user interface.
{\ltt{}atmc} repeats the following (until control-C).

It prompts the user ``{\ltt{}a) deposit, b) withdraw or c) summarize: }".
It reads the users response (which will be a, b or c).

If the user selects (a) it asks how much the user wants to deposit,
then it will ask for the account number.
It then calls the {\ltt{}deposit} procedure of {\ltt{}atmd} with the amount
of the deposit and the account number.

If the user selects (b) it asks how much the user wants to withdraw.
It then calls the {\ltt{}withdraw} procedure of {\ltt{}atmd} with the 
number the user wants to withdraw,
then it will ask for the account number.
It then calls the {\ltt{}withdraw} procedure of {\ltt{}atmd} with the amount
and the account number.
It will print the number it was actually able to withdraw (i.e., the number
returned by {\ltt{}withdraw}).

If the user selects (c) it calls the {\ltt{}summarize} procedure of 
{\ltt{}atmd}.
It will print the numbers in the array returned by {\ltt{}summarize}.
\bye
