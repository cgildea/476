\input macros
\input cstuff
\nopagenumbers
\headline{{\bf CECS 472\hfill Project 8 \hfill Spring 2002}}
\footline{Dennis Volper \hfill 29 April 2002 (Week 13 Lecture 1)\hfill
Due: 15 May 2002 2000 (Week 15 Lecture 2)}
\parindent 0pt

% Note, previously issued Week 13 Lecture 1

Purpose: Explore Routing, Domain Name Service and Subnetting Utilities

Submission: 
For the network topology question you will submit a picture which may be hand 
drawn. 
For all other you will submit short answers.
Remember, your analysis of the output of the command is an important part
of the assignment.

1)
Data link layer.
(Manual entries: arp, ping,)

Give the arp table for jaguar.
You must include the arp entries for cheetah, puma, and lab07.

ping is used only to make sure the arp table cache has the entries you want.

\vfill
2) Routing tables. (Manual entries: netstat, ifconfig)

a) What cables is the machine connected to.
b) What are the IP numbers of its interfaces to each of those cables.
c) What gateways does the machine know about.

Answer questions (a), (b) and (c) for each of the following machines:
\hfil\break
i) cheetah
\hskip1in
ii) jaguar
\hskip1in

\vfill
3) Interface configuration. (Manual entries: ifconfig, netstat)

For each IP interface give the following:
a) The IP number (from previous question).
b) The name of the interface
c) The netmask.
d) The broadcast address

netstat is listed here only because it is used
to determine the names of the interfaces.

Answer questions (a), (b), (c) and (d) for each of the following machines:
\hfil\break
i) cheetah
\hskip1in
ii) jaguar
\hfil\break

\vfill
4) Remote access of data.
(Manual entry: snmpnetstat, snmpstatus)

The College of Engineering gateways have multiple interfaces.
One of these gateways has an interface with the IP address 134.139.147.65,
another is 134.139.4.1.
a) For each of these gateways list the type of gateway and the
number of interfaces that are ``down" (this is the status of
the gateway).
b) For each of these gateways list
the interfaces are currently active on these gateways
are their internet numbers.
Notes: The 134.139.147.65 gateway uses the word Port\_xx to designate 
interfaces,
where as cheetah uses eth0 and eth1 and the
gateway 134.139.4.1 uses designations such as et.1.2.
{\ltt{}cheetah} supports snmpnetstat and snmpstatus.
The gateway 134.139.4.1 will not report out it's routing table,
but it will respond to the {\ltt{}-i} option of {\ltt{}snmpnetstat}.

Remember: each port has {\it one} IP number, obviously
one of the ports has the IP number given above.

5) Network topology. 
(Manual entries: traceroute)

Draw a map of the ``interesting" part Internet.
Include the following machines in your map and include all gateways
that connect those machines.
Your map should include IP numbers of as many of the network connections
as possible.

\medskip

{\obeylines\parskip=0pt

puma.net.cecs.csulb.edu
jaguar.cecs.csulb.edu
cheetah.cecs.csulb.edu
aardvark.cecs.csulb.edu
unix77.cecs.csulb.edu
linux157.cecs.csulb.edu
heart.cecs.csulb.edu
lab71.net.cecs.csulb.edu
lab07.net.cecs.csulb.edu
linux30.cecs.csulb.edu
charlotte.cecs.csulb.edu

}

\vfill\eject
6) Broadcast addresses.

Write a UDPtime broadcast client (and server).

For the server, 
modify {\ltt{}UDPtime.c}, so that (1) it runs
on your personal port, and (2) it accepts packets
addressed to the broadcast addresses.
Call it {\ltt{}myUDPtimebd.c}

For the client,
modify your {\ltt{}myUDPtime.c} so that 
(1) it is allowed to send a broadcast address
and
(2) it sends one time request packet and
loops on responses so it can catch answers from
multiple servers.
Call it {\ltt{}myUDPtimeb.c}

(You can eliminate or ignore the ftime stuff.)

The loop on responses should be a loop
that contains a select with a timeout set
to one second.
If the select exits due to the time out (remember
it returns the number of active sockets, and in
the case of a timeout this is 0),
terminate (exit) the program.

Testing:

Set up two copies of your server on different machines
in the lab, run your client and make sure it
gets both responses.
Remember, for broadcast to work, all machines must
be on the same subnet.

Submit:

The source code of your two programs.
\bye
