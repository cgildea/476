\input macros
\input cstuff
\headline{{\bf CECS 472\hfill Homework 23 \hfill Fall 2013}}
\footline{Nathan Pickrell \hfill 03 December 2013 (Week 15 Lecture 1)\hfill
Due: 05 December 2013 (Week 15 Lab 2)}
\parindent 0pt

Purpose: To gain an in-depth understanding of
CORBA.

You will build a CORBA client and server.

Your server will provide a class with three methods (procedures).
Your client will allow the user to call any of these procedures.

Overview: You are re-writing a (client/server) program as
a CORBA network application.
You will be given the original working C++ program.

{\it The Stringified Reference Program} 

You will also build a stringified version of the CORBA program.
CORBA is installed only on jaguar, panther and cougar;
so you will need to telnet one of these to do this assignment.

{\it The Repair Service}

Implement the methods found in the C++ version of the car repair program
({\ltt{}repair.cc}) as remote (CORBA) objects.
You will find a copy of the {\ltt{}repair.cc} program in
{\ltt{}~volper/classes/472/programs/corba}.
It is a single file, but has three parts within the file
the class, the principal procedure and the main program.
Break it into three files before starting the CORBA assignment.

Submit the following files (use the names specified):
\hfill\break
{\ltt{}repair_services.idl}: (the idl file), 
\hfill\break
{\ltt{}repair_services_i.cc}: (see {\ltt{}echo_i.cc} and {\ltt{}vector_ops_i.cc})
\hfill\break
{\ltt{}run_repair.cc}: (see {\ltt{}do_vectors.cc}),
\hfill\break
{\ltt{}repair_client.cc}: (see {\ltt{}vector_client.cc})
\hfill\break
{\ltt{}repair_server.cc}: (see {\ltt{}vector_server.cc} 

Do not submit the files automatically generated by the idl compiler.

Help: various aliases and paths need to be set up. To do this,
you should {\ltt{}source corba.sh}. This file is found in the
same directory as {\ltt{}repair.cc} (make a local copy for convenience).
Note the alias {\ltt{}gxx} has the correct (and complicated) invocation
of the {\ltt{}C++} compiler ({\ltt{}g++}) to correctly compile Corba programs.

\bye
