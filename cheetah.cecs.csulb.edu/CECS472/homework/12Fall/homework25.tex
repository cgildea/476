\input macros
\input cstuff
\headline{{\bf CECS 472\hfill Homework 25 \hfill Fall 2012}}
\footline{Dennis Volper \hfill 26 November 2011 (Week 14 Lecture 2)\hfill
Due: 3 December 2011 (Week 15 Lecture 1)}
\parindent 0pt

Purpose: To use IP version 6.

You will build an upgraded TCP time client/server pair
from your homework so that they use IP version 6.

Actually, nothing in {\it your} client/server has anything
to do with the version of IP that your socket uses.
In fact the only pieces of Comer's code that depend
on the version of IP are his {\ltt{}connectsock}
and {\ltt{}passivesock}.

Modify Comer's {\ltt{}connectsock} and {\ltt{}passivesock} so they
make an IPv6 connection between two machines.
You do not need to handle local host.

You have three ways to do this.
\hfill\break
1) make copies of Comer's code to {\ltt{}connectsock.c.back}
and {\ltt{}passivesock.c.bak} so you don't destroy the originals.
\hfill\break
2) Do the whole homework in a special subdirectory so you are
using copies of everything and use the ``submit" program to submit.
\hfill\break
3) modify Comer's code without makeing copies since this is the
last project, but this works only if you've finished all the other
projects.

Use your TCP client and server from homework 8 to test
your modifications to Comer's code.
No modifications to this client/server code are necessary.

Testing: {\ltt{}puma} and {\ltt{}cougar} have been set up to
have dual IP stacks. No others have been setup. So you must
run your client on one of these machines and your server
on the other.

Instructor Testing: I'll be testing in a special subdirectory
with my copies of the time server and client.

Handling localhost (something you do not need to do):
The easiest way to handle this is to use the fact that
if the "node" (i.e., hostname) is the {\ltt{}NULL} pointer, {\ltt{}getaddrinfo}
will return the loop back address.
The simplest way to do this is to go into the main program
and modify Comer's switch so that it says:
\hfill\break
{\ltt{}char* host=NULL;}
\hfill\break
instead of:
\hfill\break
{\ltt{}char* host="localhost";}
\hfill\break
The more complicated way is to modify {\ltt{}connectsock} to
do a {\ltt{}strcmp} on {\ltt{}host} and use {\ltt{}NULL} if it is
equal to {\ltt{}"localhost"}.
The most complicated code would be to modify connect 
so it tries IPv6 first and use IPv4 if it is not available.
Servers are allowed to bind both IPv4 and IPv6 to the same port so
a 4/6 server could be built.

\bye
