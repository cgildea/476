\footline{{\rm Dennis Volper -- 27 August 2002 \hfill}}
\centerline{{\bf SYLLABUS --- CECS 476}}
\vskip 5pt
{\obeylines\parindent 0pt
{\bf System and Network Administration}\hfill Fall 2002
Class: VEC 402 (MW 4:00--4:50PM) \hfill Course code: 11168 \hfill Lab: ECS 413 (MW 5:00--6:15PM)

Instructor: Dennis Volper \hfill Lab/Office Hours: MW 3:00PM-4:00PM

}
 
\vskip 5pt
\centerline{\bf Course Objectives}
 
To introduce the principles behind a multiuser operating system with network
access.
To teach the of administration of computers with such operating systems.
The course will cover both how the Unix system works
and the tasks the system administrator must perform.

\vskip 5pt
\centerline{\bf Prerequisites }

Some knowledge of Unix and operating systems (e.g., CECS 326).
The ability to read technical material (manuals, documentation).
Some experience in computing (several upper division courses).
 
\vskip 5pt
\centerline{\bf Books }
 
{\it Unix System Administrators Handbook}
by Nemeth et.al.
This book covers several versions of Unix, including Linux.
It covers many topics, but sometimes doesn't cover them
as completely as you would like.

{\it Running Linux}
by Welsh et.al.
This book contains information specific to Linux.
On some topics it has more information than Nemeth,
but it does not cover as many topics.

In addition you may find books from the following list helpful. 
Most can be found cheaply at major
computer stores or ordered on-line.

% {\it Linux System Administration}
% {\it{}Linux Configuration and Installation}
% {\it The Linux Internet Server}, 
% {\it{}Linux Network}.
% This is a series of books from IDG.
% There is overlap, but they cover the topic completely.

{\it Linux System Administration Handbook}
by Komarinski and Collett. 
Good in the networking area, 
but a little shallow in non-networking places.

% {\it Unix System Administrator's Edition} from SAMS publishing.
% Good, but covers all versions of Unix so the Linux information
% is scattered throughout the book.

{\it Essential System Administration} by Aeleen Frish (O'Reilly).
Good, but aimed at non-Linux versions of Unix.

{\it Linux Network Administrator's Guide} by Olaf Kirch (O'Reilly).
Very good, but only covers the networking part of Linux.

\vskip 5pt
\centerline{\bf Other Materials}

You will be issued accounts on the instructional subnet for CECS.
Online manuals will be available using those accounts.
How-To files are available for many functions.
Additional materials will be available online.
% the first set by being handed out, later sets through the copy center 
% of the campus book store.
Copies of the lecture slides are available in the book store.
Source for the lecture slides is available online in TeX format.

\vskip 5pt
\centerline{\bf Structure of the Course}
 
There will be a large number (approximately one per lab) of computer 
homework assignments, one midterm, and a final.
The assignments will be worth 40\% of the grade,
each exam will be worth 30\% of the grade.
I recommend maintain notebook documenting those actions
you have taken as a system administator.
The exams will be open note, so the better organized your notebook
the more helpful you will find it during the exams.

\vskip 5pt
\centerline{\bf Laboratory}

Your working environment for System Administration is part of the CSULB network.
There are many machines on this network, they are accessible in
many different ways.
The lab will be CECS 413, it contains the machines reserved for this course,
as well as machines for 472 and other courses.
These machines are accessible over the network,
but in a semi-secured fashion.
You may login to these machines from other rooms
and through the dial-ups, but in general, you need to login to {\tt cheetah}
first, then telnet to your Linux machine.
We recommend you perform your actual system administration in the laboratory
during lab hours.
You will be issued system administration priviledges;
exercise these privileges only for the purposes described in the assignments.

\vskip 5pt
\centerline{\bf Course Summary}

In general, we will cover a set of principles used by the Unix operating system,
you will look at the operating system to observe details on these principles,
then you will perform system administration tasks based on these principles.
Much of the material for your work will come from the on-line documentation.

\vskip 5pt
Final Exam: 
5:00PM--7:00PM, Wednesday, 16 December 2002

{\parskip=0pt

\def\week#1{\par\hangindent 0.7in {\indent\llap{\bf #1 \enspace}
\ignorespaces}}

}
\bye
