\footline{{\rm Dennis Volper -- 4 September 1996 \hfill}}
\centerline{{\bf SYLLABUS --- CECS 476}}
\vskip 6pt
{\obeylines\parindent 0pt
{\bf System and Network Administration}\hfill Fall, 1996
Class: ECS 302 (MW 7:30--8:20PM) \hfill Course code: 17061 \hfill Lab: ECS 412 (MW 8:30--9:45PM)

Instructor: Dennis Volper (Office ECS 532, x51529)\hfill Office Hours: MW 3:30PM-4:20PM, 10:00PM-11:00PM

}
 
\vskip 6pt
\centerline{\bf Course Objectives}
 
To introduce the principles behind a multiuser operating system with network
access.
To teach the of administration of computers with such operating systems.
The course will cover both how the Unix system works
and the tasks the system administrator must perform.

\vskip 6pt
\centerline{\bf Prerequisites }

Some knowledge of Unix (e.g., 126, 174, 326, 472).
The ability to read technical material (manuals, documentation).
Some experience in computing (several upper division courses).
 
\vskip 6pt
\centerline{\bf Books }
 
The text for this course is 
{\it Linux System Administrator's Survival Guide}
by Tim Parker.
{\it Unix Made Easy} (optional text for CECS 174) may be helpful
for those needing more familiarity with the Unix operating system.

\vskip 6pt
\centerline{\bf Other Materials}

You will be issued accounts on the instructional subnet for CECS.
Online manuals will be available using those accounts.
How-To files are available for many functions, 
these may be found on the CDROM attached to the text book.
Additional materials will be available online.
The lecture slides will be made available, probably by being handed out.
Source for the slides is available online.

\vskip 6pt
\centerline{\bf Structure of the Course}
 
There will be a large number of computer homework assignments, 
one midterm, and a final.
The assignments will be worth 40\% of the grade,
each exam will be worth 30\% of the grade.
     
\vskip 6pt
\centerline{\bf Laboratory}

Your working environment for System Administration is part the CSULB network.
There are many machines on this network, they are accessible in
many different ways.
Some parts of an assignment may specify the use of specific machines, others
may allow you to choose your machine.
The lab will be CECS 412, it contains the machines reserved for this course,
as well as machines for 472 and the Ada courses.
These machines are accessible over the network,
so you may login to these machines from other rooms
and through the dial-ups.
We recommend you perform your actual system administration in the laboratory
during lab hours.

\vskip 6pt
\centerline{\bf Course Summary}

We will cover various chapters from Parker's book.
We will not cover the book chapters in order.
Since this is the first offering of this course,
plans are subject to change depending on what works.

In general, we will cover a set of principles used by the Unix operating system,
you will look at the operating system to observe details on these principles,
then you will perform system administration tasks based on these principles.
Much of the material for your work will come from the on-line documentation.

%\vskip 6pt
%Final Exam: 
%
%5:00--7:00PM, Monday, 16 December 1996
%
{\parskip=0pt

\def\week#1{\par\hangindent 0.7in {\indent\llap{\bf #1 \enspace}
\ignorespaces}}

}
\bye
