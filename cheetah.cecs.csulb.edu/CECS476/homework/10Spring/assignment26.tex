\input macros
\rightskip=0pt plus 1fill
\input cstuff
\headline{{\bf CECS 476\hfill Assignment 26 \hfill Spring 2010}}
\footline{Dennis Volper \hfill 29 April 2010 (Week 13 Lecture 2)\hfill 
Due: 4 May 2010 (Week 14, Lab 1)}
\parindent 0pt

This project is about dhcp and remote booting.

You dhcp server is jaguar.
You will examine the setup of the dhcp server on jaguar for this project.

Be careful that the you submit answers to the questions are correct
for the machine you administer.

Examine the {\ltt{}dhcpd.conf} file and answer the following questions.
What file does your machine use to boot?
What domain name server does your machine use?
Give the lines that set the subnet, netmask and router for
your machine.
Give the lines that set the name and IP address of your machine.

Search the {\ltt{}rc.d} directory. 
Which file starts {\ltt{}dhcp}?


The file from the previous questions also starts the trivial file
transfer protocol server.
It is started with an option, what does that option mean?

Examine the {\ltt{}in.tftpd} process on jaguar.
What option is it running with?
What does that option mean?

Examine the {\ltt{}in.tftpd} entry in {\ltt{}inetd.conf}.
What options does it start {\ltt{}in.tftpd} with?
What do these option mean?

Which is more secure, the {\ltt{}inetd.conf} or the way
the instructor actually started it?
(This is what happens when you get in a hurry.)

Examine the {\ltt{}default} config file {\ltt{}pxelinux.cfg}.
How many boot options are there?
(Hint: Did you see boot options when you did the install.)

Examine the {\ltt{}default.std} config file {\ltt{}pxelinux.cfg}
(this file is not used).
How many boot options are there?
Which kernels can be booted?
Which message files can be displayed?
\bye
