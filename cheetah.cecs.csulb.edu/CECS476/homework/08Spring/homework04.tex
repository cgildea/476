\input macros
\rightskip=0pt plus 1fill
\input cstuff
\headline{{\bf CECS 476\hfill Homework 4\hfill Spring 2008}}
\footline{Dennis Volper \hfill 6 February 2008 (Week 2 Lecture 2)\hfill 
Due: 11 February 2008 (Week 3 Lab 1)}
\parindent 0pt

On all assignments, be sure to indicate the name of the
machine you are assigned to administer.

This first question of this assignment emphasizes shell programming,
the remaining questions have you examine the file system.

Make sure the {\ltt{}bob} exists on your 
machine.

As bob, write a shell script that examines all the processes on the
system and ``reports" c-shell processes
into a file called {\ltt{}cshlog}.
First, this script should append to the {\ltt{}cshlog} the {\ltt{}date}.
Then, it should do a {\ltt{}ps aux}, and all lines (use {\ltt{}grep})
containing the letters csh but not containing the letters cpeek
should be appended to the {\ltt{}cshlog}.
Your script file should be called {\ltt{}cpeek}

Test your shell script by
running your command several times with various users logged 
in (for example root and your csv account)
(i.e., just make sure it works).

1) Report: the exact contents of {\ltt{}cpeek}.
(Do NOT report the contents of your cshlog.)

On the {\ltt{}cheetah} examine and report the following:

2) What three hard drives are attached to the file tree and where (mount/df)?

3) On the root ({\ltt{}/}) file system, how much disk is available (df)?

4) On the root ({\ltt{}/}) file system, what is the file system type
and is the file system read only or read/write (mount)?

5) In your home directory, how much space have you used (not much yet) (du)?

6) Your home directory is on one of the hard drive partitions.
Report the line in the {\ltt{}fstab} that causes that partition
to be mounted.

On the system you administer, using the system administrator account;
examine the superblock of the linux partition on your hard drive
({\ltt{}/dev/hda1}) using the {\ltt{}dumpe2fs} command.

7) Report: The file system state, the block size and the number of groups
in that file system.

8) {\bf fdisk -l}:
For the disk you administer describe the partitions that are on the disk?
For each partition {\it report}, what cylinder it starts at,
what cylinder it ends at,
and what kind of a partition it is (Linux, DOS, swap).
Be very careful with this one, 
DO NOT MODIFY or WRITE anything with fdisk; 
you could erase your hard drive.

\bye
