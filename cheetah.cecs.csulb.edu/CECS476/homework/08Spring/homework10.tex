\input macros
\rightskip=0pt plus 1fill
\input cstuff
\headline{{\bf CECS 476\hfill Homework 10 \hfill Spring 2008}}
\footline{Dennis Volper \hfill 27 February 2008 (Week 5 Lecture 2)\hfill 
Due: 2 March 2008 (Week 6 Lab 1)}
\parindent 0pt

Purpose: This assignment covers backups and termcap.

{\bf Backups---}

{\bf tar:}
Make a tar back up of the home directory of your user (bob) to your floppy
disk.
Confirm that the backup was made using the {\ltt{}t} option.
Note: This is a tar floppy, it does not have a directory structure,
so no mkfs should be done; all information on the floppy before
the tar will be destroyed. After the you do the backup, your
floppy looks like a tape.

1) Submit: the sequence of commands from login up through the
tar command that lists the tar contents.

{\bf tar, extracting an archive:}
Login as root, this puts you in {\ltt{}/root}.
Extract the contents of the compressed tar file {\ltt{}~djv/giant.tgz},
preserving the permissions.
(Note, this should build a directory called {\ltt{}giant} inside {\ltt{}/root}.)

2) Submission: The exact command you used.
Inside {\ltt{}giant} there are three subdirectories, submit: the names of
these three directories. (This lets me know you unwound the archive.)
Pick any of these subdirectories, you will find a file inside it,
submit: the name of the file, who owns the file, what group the file
belongs to and the access rights (the {\ltt{}rwx} stuff) for the file.
(This lets me know you preserved permissions.)

{\bf Terminals---}

3) Termcap.
On the linux console: what sequence is used to 
start underlining?
%clear from the current cursor position to the end of line?

4) stty.
On the linux console: what key is used to stop (pause) the output to the
screen?
What command would you give to define control-Y to be the key used
to stop the output?

\bye
