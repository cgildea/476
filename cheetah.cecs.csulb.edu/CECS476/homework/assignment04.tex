\input macros
\rightskip=0pt plus 1fill
\input cstuff
\headline{{\bf CECS 476\hfill Assignment 4\hfill Spring 2014}}
\footline{Nathan Pickrell \hfill 30 January 2014 (Week 2 Lecture 2)\hfill 
Due: 4 February 2014 (Week 3 Lab 1)}
\parindent 0pt

On all assignments, be sure to indicate the name of the
machine you are assigned to administer.

This first question of this assignment emphasizes shell programming,
the remaining questions have you examine the file system.

Make sure the {\ltt{}bob} exists on your 
machine.

Simple shell script.

As bob, write a shell script that examines all the processes on the
system and ``reports" c-shell processes
into a file called {\ltt{}cshlog}.
First, this script should append to the {\ltt{}cshlog} the {\ltt{}date}.
Then, it should do a {\ltt{}ps aux}, and all lines (use {\ltt{}grep})
containing the letters csh but not containing the letters cpeek
should be appended to the {\ltt{}cshlog}.
Your script file should be called {\ltt{}cpeek}

Test your shell script by
running your command several times with various users logged 
in (for example root and your csa account)
(i.e., just make sure it works).

1) Report: the exact contents of {\ltt{}cpeek}.
(Do NOT report the contents of your cshlog.)

More advanced shell script.

As bob, write a shell script called {\ltt{}clook}.
This shell script takes an argument that should be the name of the directory.
If it is given the name of something that isn't a directory (or doesn't
exist) it should print ``Sorry".
If it is given the name of a directory it should print the names of
any items in that directory that are files and are executable. 

Test your shell script by running your command on {\ltt{}/etc/printcap}
(it's a file and should print ``Sorry") and on {\ltt{}~vjd} (do an {\ltt{}ls}
to see confirm the names of the executables.

2) Report: the exact contents of {\ltt{}clook}.

On the {\ltt{}cheetah} examine and report the following:

3) What three hard drives are attached to the file tree and where (mount/df)?
Linux lists one hard drive as a softlink called {\ltt{}/dev/root}.
You can determine which hard drive this is by doing an {\ltt{}ls -l} on
this link.

4) On the root ({\ltt{}/}) file system, how much disk is available (df)?

5) On the root ({\ltt{}/}) file system, what is the file system type
and is the file system read only or read/write (mount)?

6) In your home directory, how much space have you used (not much yet) (du)?

7) Your home directory is on one of the hard drive partitions.
Report the line in the {\ltt{}fstab} that causes that partition
to be mounted.

On the system you administer, using the system administrator account;
examine the superblock of the linux partition on your hard drive
({\ltt{}/dev/sda2}) using the {\ltt{}dumpe2fs} command.

8) Report: The file system state, the block size and the number of groups
in that file system.

9) {\bf fdisk -l}:
For the disk you administer describe the partitions that are on the disk?
For each partition {\it report}, what cylinder it starts at,
what cylinder it ends at,
and what kind of a partition it is (Linux, DOS, swap).
Be very careful with this one, 
DO NOT MODIFY or WRITE anything with fdisk; 
you could erase your hard drive.

\bye
