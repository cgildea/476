\input macros
\rightskip=0pt plus 1fill
\input cstuff
\headline{{\bf CECS 476\hfill Project 16 \hfill Spring 2002}}
\footline{Dennis Volper \hfill 3 April 2002 (Week 9 Lecture 2)\hfill 
Due: 8 April 2002 (Week 10, Lab 1)}
\parindent 0pt

This assignment covers the boot/root disks, and what is on the release CD.

\bigskip
1) {\bf fdisk}:
For the disk you administer describe the partitions that are on the disk?
For each partition {\it report}, where it starts, where it ends, and what kind
of a partition it is (Linux, DOS, swap).
Be very careful with this one, you have to be root to run fdisk,
DO NOT MODIFY or WRITE anything with fdisk; 
you can easily erase the disk.
Record the information very carefully, you have to use this information
to rebuild after your disk crash.

2) {\bf pkgtool}:
What packages are installed on the machine you administer.
{\it Report} just the first 6. Don't bother to record the others, you will
be told what to install in the instructions for the install project.

\bigskip
3) {\bf boot sets}:
For each of the following configurations select a boot disk.
For each (a,b,c) {\it report} which boot disk you would select.
Assume that your machine has both the equipment listed as "To" and the
equipment listed as "from".
List one of the disks found on the CD rom that came with the book.
For your convenience, my copy of the CD is mounted on {\ltt{}jaguar} under
{\ltt{}/slak8}. 
One of the following configurations is {\it not} listed in the book,
you {\it must} read the documentation on the CD to determine the correct
boot disk.

a) Install to an Adaptec 1542 SCSI controller (and SCSI disks) from another
machine on the network.

b) Install to an IDE hard drive from floppy disks. (ouch!)

c) Install to an IDE hard drive from a Sanyo CD-ROM.

\bigskip
4) {\bf boot disks}:
Build two 3.5" 1.44Mdisks from the appropriate files from the Linux CD.
One of these disks is to be a boot disk.
It must contain a kernel that supports IDE drives
and networking ({\ltt{}net.i}).
The other should be a root disk ({\ltt{}color.gz}).

You can find the disks under the {\ltt{}~djv}
directory (visible from the machine you administer).

{\it Report:} the exact commands you used to build the disks.

Test: Making sure that all other users are off the machine,
place your boot disk in the A: drive and
reboot the machine using the shutdown command.
Follow the instructions, paying careful attention for when to swap disks.
Once you have booted from floppy, examine the {\ltt{}/bin} directory
(see next report section).
Remove your disks from the drive and reboot using the shutdown command.

{\it Report:} the number of executables you found in {\ltt{}/bin}.
(Note: some of the things in {\ltt{}/bin} are not executables, they are links.)

\bye
