\input macros
\rightskip=0pt plus 1fill
\input cstuff
\headline{{\bf CECS 476\hfill Project 9 \hfill Spring 2002}}
\footline{Dennis Volper \hfill 25 February 2002 (Week 5 Lecture 1)\hfill 
Due: 27 February 2002 (Week 5, Lab 2)}
\parindent 0pt

This assignment covers building a new system kernel.
WARNING: this may take a while.
WARNING: be careful, you can easily erase your operating system.

Build and boot with a new (trimmed down) installation of the kernel.

A) Get a copy of the {\ltt{}config.net.i} configuration file from
the {\ltt{}~djv} directory. 

I recommend you compile and install to make sure it works.
(I.e., skip B for now and compile and install a "stock" kernel to
make sure you know the procedure).

B) Configure the kernel. 

Eliminate any unnecessary drivers and/or components so that your kernel is
as small as possible.

For your information your machine has:

an IDE hard drive, a mouse (usually PS/2), a PCI bus, a 
a network card, (the exact type varies)
a video card (some are ATI and others are generic SVGA) and a floppy drive.

(WARNING: double check using dmesg what type of network card your
machine has and look at the back of your computer to see
if you have a serial (DB connector) or PS/2 (DIN connector) mouse.)

C) Build a new kernel ({\ltt{}bzImage}).

1) Report: the elapsed time to do the build.

Note: do a date before you start, write it down and look at
the date stamp on the {\ltt{}bzImage} file (which is created at
the end or the build).

D) Modify the {\ltt{}lilo.conf} file to have a second linux
bootable configuration.
Duplicate the 4 lines and two comment lines (total of 7 lines),
change the image to {\ltt{}bzImage} and the label to {\ltt{}bz}.
Make sure the original bootable configuration entry stays first so that
it is the default.

E) Move the new kernel so that it is {\ltt{}/bzImage}.
(Do NOT under ANY circumstances remove or replace {\ltt{}vmlinuz}

F) Run lilo, it should show two ``Added" reports.

G) Reboot, shift key down; select bz.

2) Report: the contents of {\ltt{}/proc/sys/kernel/version} 

3) Report: the size of you new kernel.

\bye
