\input macros
\rightskip=0pt plus 1fill
\input cstuff
\headline{{\bf CECS 476\hfill Project 11 \hfill Spring 2002}}
\footline{Dennis Volper \hfill 4 March 2002 (Week 6 Lecture 1)\hfill 
Due: 6 March 2002 (Week 6, Lab 2)}
\parindent 0pt

This project is about general networking.
You will be asked about existing configurations. 
In general the answers to the questions can be found on the machine for
which you are an administrator.
No root priviledges should be necessary to answer any of the questions.

Questions about network drivers.
(visit {\ltt{}driver/net} in the kernel source code)

1) Support for several types of 3-Com (3cXXX) cards are available
for the unix kernel, list 4 such cards.

2) The etherworks 3 driver supports what three cards?

3) On a Parallel line cable running PLIP, what is pin 25 used for?

Questions about configuration of TCP/IP.
(examine the front of your machine, also run the ifconfig and netstat commands)

4) Use a command to find out the official hostname of your machine?
What is the official hostname and what command did you use to find this out?

5) What is the Hardware (ethernet address) of your machine?

6) What does it use for its broadcast address?

7) The routing table for your machine contains one non-local entry.
What is that entry?

8) What netmask does the loopback interface use?

Question about arp.  Ping two other of the lab machines. 

9) What entries are in the arp table?
Give the full entries including the hardware addresses.

Questions about The resolver. 
Examine {\ltt{}/etc/resolv.conf} and {\ltt{}/etc/nsswitch.conf}

10) What methods (NIS, DNS/bind, hosts file) does your machine use
to locate a hostname and in what order?

11) What machine does your machine use for DNS (resolver)?

12) Your machine allows certain other machines to be accessed without
typing the entire (fully qualified) host name. Which hostnames can
be accessed using this shorthand.
\bye
