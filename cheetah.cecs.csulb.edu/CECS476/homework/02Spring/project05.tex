\input macros
\rightskip=0pt plus 1fill
\input cstuff
\headline{{\bf CECS 476\hfill Project 5 \hfill Spring 2002}}
\footline{Dennis Volper \hfill 11 February 2002 (Week 3 Lecture 1)\hfill 
Due: 13 February 2002 (Week 3, Lab 2)}
\parindent 0pt

Purpose: This assignment covers disk and file system administration.

Submission: You will submit a written document for your group.

1) On the machine you administer there is an unused 
partition.
Build an ext2 file system on that disk or partition and attach
that file system as {\ltt{}/newdisk}.

1a) Warning: you will need to run {\ltt{}fdisk} to make sure the parition
is marked as ``linux native".

Be careful not to damage the existing
Linux partitions on your drive or you will crash your system.

Submit: a brief description what you did with {\ltt{}fdisk} and
a listing of your partition table.
If your machine has two disks list the partition tables of both your disks.

1b) Use {\ltt{}mke2fs} to build an ext2 file system in your new partition.
Try the mount by hand (this is the safest way to make sure it is set
up right).

Submit: the disk free {\ltt{}df} report on the new file system.

1c) Modify your system so that the mount will 
occur automatically at boot.

Submit: the line you added to your {\ltt{}fstab}.

{\bf Bring in one already formated DOS floppy.}

2) On your formatted, set-up a ext2 file system,
Make an entry in {\ltt{}fstab} that allows any user insert and 
mount an ext2 floppy using {\ltt{}/mnt} as the mount point.
Change the permissions on {\ltt{}/mnt} to 777 ({\ltt{}chmod 777 /mnt}).
Login as yourself or bob (not root), put your ext2 floppy
in the drive,
mount it, copy a file to the floppy, unmount and remove the floppy.

Submit: the line you entered into {\ltt{}fstab} and
the exact mount command you used.

3) As the system administrator: run an {\ltt{}fsck} on your floppy. 
Unix wants the file system being checked to be unmounted, so it is not 
changing while the check is performed.

Submit: the exact command you used and summarize what the command output.

Clean up: change the permissions on {\ltt{}/mnt} back to 755.

\bye
