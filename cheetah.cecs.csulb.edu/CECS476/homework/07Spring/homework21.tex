\input macros
\rightskip=0pt plus 1fill
\input cstuff
\headline{{\bf CECS 476\hfill Homework 21 \hfill Spring 2007}}
\footline{Dennis Volper \hfill 18 April 2007 (Week 11 Lecture 2)\hfill 
Due: 20 April 2007 (Week 12, Lab 1)}
\parindent 0pt

This project is about network news.

A ``limited" network news is running on jaguar.
It is limited in the following sense
1) there are only two news groups,
2) it is only willing to share (feed) these news groups with (to) the CECS 476
machines,
3) jaguar accepts newsfeeds only from the CECS 476 machines.
In other words, we have an isolated, two news group world to play with.

Jaguar only permits news readers from jaguar to read news.
This means that (for example) from {\ltt{}lab77} you cannot
read news using the {\ltt{}innd} on jaguar.
However, jaguar is willing to exchange news with {\ltt{}lab77}.
That means that the {\ltt{}innd} on {\ltt{}jaguar} is willing to transfer
news to and from the {\ltt{}innd} on {\ltt{}lab77}.

If you followed the install instructions;
there is no {\ltt{}innd} running on the machine you are assigned to administer,
but you did install it.
Your job is to configure and start the news service for your machine.
This job includes starting {\ltt{}innd}.
You should be able to read and post news to your local machine.
Your local machine should exchange news with {\ltt{}jaguar}.
News posted on your machine should be sent to {\ltt{}jaguar}.
News that {\ltt{}jaguar} has should be feed to your machine.

Jaguar has two news groups, {\ltt{}local.476a} and {\ltt{}local.476b}.
Get them both to work.
Actually, if you get one news group to work, the other is simple.

Step 1. Use news on {\ltt{}jaguar}.
To become familiar with news I recommend that you log into {\ltt{}jaguar},
read the messages posted in the news group.
A reasonable news reader is {\ltt{}trn},
but if you prefer {\ltt{}tin} you are welcome to use it.
In addition, post a message to a new group and have someone else in your group
read it to make sure it got through.
\hfill\break
{\bf Report:} no report for this step.

Step 2.
Examine the configuration files on {\ltt{}jaguar}.
I have deliberately unprotected the configuration files so anyone can read
them.
The program {\ltt{}inncheck} (if you run it) will complain about this 
because it considers this unprotecting to be a potential security hole,
but it won't effect the ability of {\ltt{}innd} to run.
Notice particularly those files which have been modified recently.
\hfill\break
{\bf Report:} no report for this step.

Step 3. Configure news (and start {\ltt{}innd}) for the machine you are
assigned to administer.
Arrange for your news groups to be transferred to/from {\ltt{}jaguar}.
Note: if someone successfully posts news on another lab machine, the posting
should become available on yours.
You should make the following three tests:
\hfill\break
1) Post news on your machine make sure you can read it
on your machine.
\hfill\break
2) Post news on your machine read it on your {\ltt{}jaguar}.
\hfill\break
3) Post news on {\ltt{}jaguar} and read it on your machine.
\hfill\break
{\bf Report:}
\hfill\break
1) What files you changed (Give the full rooted path name such as
{\ltt{}/usr/tmp/xx/yy}.
\hfill\break
2) For each of the files you changed, give the contents of that file.
Do not include the comment lines, just the active/command lines.

Notes: Use traditional spooling. On your computer, uses {\ltt{}bob}
and only {\ltt{}bob} to post news. You may also use root on
your local box to read news, do not post with root.
On jaguar, use your class account to read and post news;
do {\it not} use your class account to either read or post news on your
computer. Do not do any administrative actions on this project
while you are root; make sure you always {\ltt{}su} to news before
doing anything administatively. To start, stop and restart news
the use of the {\ltt{}rc.news} script is recommended.
If you have trouble use syslog to dump {\ltt{}news.*} into a log file
({\ltt{}jaguar} does this).

Specials:
%the install doesn't setup news quite correctly.
The inn install wasn't in the main sequence so you have to
install this package separately. using the following instructions
You need to the following while loged in as root:
%\hfill\break
%{\ltt{}rm /usr/bin/inews}
%\hfill\break
%{\ltt{}ln -s /usr/lib/news/bin/inews /usr/bin/inews}
%\hfill\break
%{\ltt{}chmod 555 ~news/bin/inews}
\hfill\break
{\ltt{}tar -xzpf ~djv/inn-2.4.2-i486-1.tgz}
\hfill\break
{\ltt{}source install/doinst.sh}

You will also have to enable the sending of news (it's not the default)
\hfill\break
{\ltt{}chmod a+x /usr/lib/news/bin/inews}

\bye
