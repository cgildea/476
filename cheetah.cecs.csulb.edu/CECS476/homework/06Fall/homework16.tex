\input macros
\rightskip=0pt plus 1fill
\input cstuff
\headline{{\bf CECS 476\hfill Homework 16 \hfill Fall 2006}}
\footline{Dennis Volper \hfill 25 October 2006 (Week 9 Lecture 2)\hfill 
Due: 30 October 2006 (Week 10, Lab 1)}
\parindent 0pt

This assignment covers the boot/root disks, and what is on the release CD.

\bigskip
1) {\bf fdisk}:
For the disk you administer describe the partitions that are on the disk?
For each partition {\it report}, where it starts, where it ends, and what kind
of a partition it is (Linux, DOS, swap).
Be very careful with this one, you have to be root to run fdisk,
DO NOT MODIFY or WRITE anything with fdisk; 
you can easily erase the disk.
Record the information very carefully, you have to use this information
to rebuild after your disk crash.

2) {\bf pkgtool}:
What packages are installed on the machine you administer.
{\it Report} just the first 6. Don't bother to record the others, you will
be told what to install in the instructions for the install project.

\bigskip
3) {\bf boot sets}:
Examine the slackware cd image on {\ltt{}jaguar}.
The howto describes several boot sets.
What is the purpose of the {\ltt{}speakup.s} boot set.

\bigskip
4) {\bf root sets}:
Exam CDROM number 3 (disk3) on {\ltt{}jaguar}.
If you use a boot floppy;
what root floppies would also be required by the boot (and
by the install).
List those disks. Note: {\ltt{}color.gz} does not really
exist under this method of booting.

\bigskip
5) The installation CDROM available as a using ``boot from network".
Boot the installation CDROM from the network. (I want to make sure
you can do this {\it before} we erase your harddrive.)

Report the number of entries you found in {\ltt{}/bin}.
(This answer doesn't have any importance, it's just to make sure
that you've been able to do the boot.)

\bye
