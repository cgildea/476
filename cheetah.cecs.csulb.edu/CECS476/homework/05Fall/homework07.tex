\input macros
\rightskip=0pt plus 1fill
\input cstuff
\headline{{\bf CECS 476\hfill Homework 7 \hfill Fall 2005}}
\footline{Dennis Volper \hfill 21 September 2005 (Week 4 Lecture 2)\hfill 
Due: 26 September 2005 (Week 5 Lab 1)}
\parindent 0pt

Purpose: This part of the assignment covers init.

Submission: You will submit a written document, plus a ``demonstration".

You want to attach a terminal to serial port 0 (DOS calls this COM1).
(1) Report: What line would you need to change in the inittab.
(Do not make this change, just report the line that would need
to be changed.)

The following must be done from CONSOLE:

Modify your {\ltt{}inittab} so that control-alt-delete does not
reboot the machine.
Cause {\ltt{}init} to re-read {\ltt{}inittab}.
Check to make sure that your have successfully disabled control-alt-delete.

Good practice, never delete or modify a default configuration line, 
comment the old one out (so you can go back to it) and add the new one.

{\it Demonstration}: 
call the instructor over and let him type control-alt-delete
(your machine should not reboot).

Cleanup: restore {\ltt{}inittab} to it's original condition. 

(2) Report: The exact text of the line that disables control-alt-delete.
The command you ran to make that line take effect

On shutdown, how many seconds does {\ltt{}init} wait before 
forcibly terminating processes? (Hint: {\ltt{}man init})
\break
3) Report: the number of seconds.

This part of the assignment cron and at.

Cron: as bob; set up you crontab so that you run an {\ltt{}ls} command
every minute (redirect the output of the {\ltt{}ls} to a file). 

4) Report: your crontab entry. Just the one line is ok.

Clean up: remove your crontab entry.

At: Set up an at entry, that in 5 minutes from now will
run the {\ltt{}ls} command.
(Again, redirect the output of the {\ltt{}ls} to a file.) 

5) Report: the exact command you used to do this.

\bye
