\input macros
\rightskip=0pt plus 1fill
\input cstuff
\headline{{\bf CECS 476\hfill Homework 23 \hfill Fall 2007}}
\footline{Dennis Volper \hfill 3 December 2007 (Week 14 Lecture 1)\hfill 
Due: 5 December 2007 (Week 14, Lab 2)}
\parindent 0pt

This homework is about setting up Domain Name Service.

Set up the files for the machine you administer
to be a DNS name server. 
You will claim to be a primary server for the domain netXX.cecs.csulb.edu 
where XX is the number of your machine.
(For example those of you on lab28 will set up as a primary
for the domain net28.cecs.csulb.edu.)
You will also claim to be a primary
server (reverse lookup) for the 192.168.1.0 subnet.

To make this project simpler, you domain has only two machines.
Your DNS files should have name/number and reverse mappings
for the machine {\ltt{}test01.netXX.cecs.csulb.edu} 
with a number of {\ltt{}192.168.1.1}
and for the 
for the machine {\ltt{}test02.netXX.cecs.csulb.edu} 
with a number of {\ltt{}192.168.1.2}
You will need to build {\ltt{}named.conf},
{\ltt{}named.local} and the forward and reverse files.
You should use a copy of an existing {\ltt{}root.cache}.
You should place {\ltt{}named.conf} in {\ltt{}/etc}
and all other files in {\ltt{}/etc/named}.
When you have the files set up, you should start {\ltt{}/usr/sbin/named} 
(the program), by hand.

To prevent everything from crashing if you goof up your configuration,
I recommend you do not change your {\ltt{}resolv.conf}.
This means that ``normal" name look-ups will still go to {\ltt{}cheetah}.
The only time you will use your server is if you specify it or
to specifically switch to your it.

Testing: Use an {\ltt{}nslookup} or {\ltt{}dig} or {\ltt{}host}
specifically specifying your machine as the server and look up
{\ltt{}test01.netXX.cecs.csulb.edu} and make sure it finds the IP number.
Repeat this test looking up the IP number ({\ltt{}192.168.1.1})
and make sure it finds the name.
Do this for the other name and address as well.

{\bf Report:} Which files did you install and what was in them.
Suggestion: give me a printout of each of the files. As the first or second
line in the file put a comment that contains the name of the file.

{\bf Report:} Give exactly one of your tests.
This will indicate which command you used in testing and the
exact format of that command.

\bye
