\input macros
\rightskip=0pt plus 1fill
\input cstuff
\headline{{\bf CECS 476\hfill Assignment 8 \hfill Spring 2014}}
\footline{Nathan Pickrell \hfill 13 February 2014 (Week 4 Lecture 2)\hfill 
Due: 18 February 2014 (Week 5 Lab 1)}
\parindent 0pt

This assignment covers building a new system kernel.
WARNING: this may take a while.
WARNING: be careful, you can easily erase your operating system.

Build and boot with a new installation of the kernel.

A copy of the configuration file used for your standard kernel
is found in {\ltt{}/boot/config}. Use this file only
to copy it down as {\ltt{}.config}, do not modify it.

I recommend you compile and install to make sure it works.
(I.e., skip the ``eliminate" part of step A for now and compile and
install a "stock" kernel to
make sure you know the procedure).
This kernel should be almost the same size as the stock kernel and should
work exactly like the stock kernel

A) Configure the kernel. 

Desirable:
Eliminate any unnecessary drivers and/or components so that your kernel is
as small as possible.
Minimal: Eliminate at least one unnecessary kernel component so your kernel
is smaller than the stock kernel.

%Add any drivers necessary for enabling your networking.

Your machine has:
an IDE hard drive, a SATA Hard drive, a PCI bus, a PS/2 mouse
and
an Intel PRO/100 on board network adapter.

If you get into trouble in step B, copy it in as {\ltt{}.config}
and start again.

B) Build a new kernel ({\ltt{}bzImage}).

1) Report: the date stamp on the bzImage file.

C) Modify the {\ltt{}lilo.conf} file to have a second linux
bootable configuration.
Duplicate the 4 lines and two comment lines (total of 7 lines),
change the image to {\ltt{}bzImage} and the label to {\ltt{}bz}.
Make sure the original bootable configuration entry stays first so that
it is the default.

D) Move the new kernel so that it is {\ltt{}/boot/bzImage}.
(Do NOT under ANY circumstances remove or replace {\ltt{}vmlinuz})

E) Run lilo, it should show two ``Added" reports.

F) Reboot, shift key down; select bz.

Make sure your kernel is working by,
i) can bob login (and ls his home dir)
ii) can your personal account login (and ls your home dir). 

2) Report: the contents of {\ltt{}/proc/sys/kernel/version} 

3) Report: the size of you new kernel and the size of the stock kernel.

After you have done this, you may reboot to the standard kernel.
\bye
