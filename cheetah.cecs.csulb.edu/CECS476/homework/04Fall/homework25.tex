\input macros
\rightskip=0pt plus 1fill
\input cstuff
\headline{{\bf CECS 476\hfill Homework 25 \hfill Fall 2004}}
\footline{Dennis Volper \hfill 29 November 2004 (Week 14 Lecture 1)\hfill 
Due: 1 December 2004 (Week 14, Lab 2)}
\parindent 0pt

This project is about SAMBA.

1) Create a directory {\ltt{}/home/share} on your machine and
create a couple text files in it. Call them {\ltt{}sam.txt} and 
{\ltt{}joe.txt}.

2) Modify the samba configuration file so that your security is by
share and you share the {\ltt{}/home/share} directory as public
read only. Call this share {\ltt{}[myshare]}. (Yes, you'll need
to do a few other housekeeping items.)
Also, turn off the printer shares and anything you are not really
offering.

3) Set up the rc files ({\ltt{}rc.samba}) so they are executable.
(You did disabled if you followed the instructions during the install
project.) Run {\ltt{}rc.samba}.
If you have already started samba (and it didn't work right), 
{\ltt{}rc.samba} has start, stop and restart as parameters.
Once you have started the samba,
check to make sure the two samba daemons are running.
If they are not, check the log files for messages.

4) Test your configuration: Go to the NT machine in the lab.
Login as guest (it should be logged in already). Click on network 
neighborhood. If your machine shows up and when you click on
your machine's icon the share should appear, if your
machine doesn't show up, use the ``find computer" option on the pull-down
and find {\ltt{}lab}{\it{}xx} where {\it xx} is the number of your machine.
(The NT machine is configure so it doesn't need the {\ltt{}net.cecs.csulb.edu}
suffix.

Submit: a copy of your smb.conf with the comments removed.

Demonstration: When you hand your submission to me, you must use the NT machine
to demonstrate that your share works.
\bye
