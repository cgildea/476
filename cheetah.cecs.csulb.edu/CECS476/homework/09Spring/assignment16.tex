\input macros
\rightskip=0pt plus 1fill
\input cstuff
\headline{{\bf CECS 476\hfill Assignment 16 \hfill Spring 2009}}
\footline{Dennis Volper \hfill 24 March 2009 (Week 9 Lecture 1)\hfill 
Due: 26 March 2009 (Week 9, Lab 2)}
\parindent 0pt

This assignment covers the boot/root disks, and what is on the release CD.

\bigskip
1) {\bf fdisk}:
For the disk you administer describe the partitions that are on the disk?
For each partition {\it report}, where it starts, where it ends, and what kind
of a partition it is (Linux, DOS, swap).
Be very careful with this one, you have to be root to run fdisk,
DO NOT MODIFY or WRITE anything with fdisk; 
you can easily erase the disk.
Record the information very carefully, you have to use this information
to rebuild after your disk crash.

2) {\bf pkgtool}:
What packages are installed on the machine you administer.
{\it Report} just the first 6. Don't bother to record the others, you will
be told what to install in the instructions for the install project.

\bigskip
3) {\bf etherboot}:
Examine CDROM number 6 (disk26 on on {\ltt{}jaguar}.
What is a {\ltt{}etherboot}? (see the README)

\bigskip
4) {\bf dhcp/tftp}:
Examine the {\ltt{}dhcpd.conf} {\ltt{}jaguar}.
What is the name of the file that will be download using tftp to
start the boot sequence on your machine?
What is the host entry for your machine?

\bigskip
5) The installation CDROM is available using ``boot from network".
Boot the installation CDROM from the network. (I want to make sure
you can do this {\it before} we erase your harddrive.)

Report the number of entries you found in {\ltt{}/bin}.
(This answer doesn't have any importance, it's just to make sure
that you've been able to do the boot.)

\bye
