\input macros
\rightskip=0pt plus 1fill
\input cstuff
\headline{{\bf CECS 476\hfill Homework 16 \hfill Spring 2005}}
\footline{Dennis Volper \hfill 30 March 2005 (Week 9 Lecture 2)\hfill 
Due: 4 April 2005 (Week 10, Lab 1)}
\parindent 0pt

This assignment covers the boot/root disks, and what is on the release CD.

\bigskip
1) {\bf fdisk}:
For the disk you administer describe the partitions that are on the disk?
For each partition {\it report}, where it starts, where it ends, and what kind
of a partition it is (Linux, DOS, swap).
Be very careful with this one, you have to be root to run fdisk,
DO NOT MODIFY or WRITE anything with fdisk; 
you can easily erase the disk.
Record the information very carefully, you have to use this information
to rebuild after your disk crash.

2) {\bf pkgtool}:
What packages are installed on the machine you administer.
{\it Report} just the first 6. Don't bother to record the others, you will
be told what to install in the instructions for the install project.

\bigskip
3) {\bf boot sets}:
Examine the slackware cd image on {\ltt{}jaguar}.
The howto describes several boot sets.
What is the purpose of the {\ltt{}speakup.s} boot set.

\bigskip
3) {\bf root sets}:
Exam cd rom number 4 (disk4) on {\ltt{}jaguar}.
If you use the {\ltt{}bare.i} boot disk;
what root disks would also be required by the boot (and
by the install).
List those disks. Note: {\ltt{}color.gz} does not really
exist under this method of booting.

\bigskip
5) {\bf boot disks}:
Build two 3.5" 1.44Mdisks from the appropriate files from the {\ltt{}djv} home directory.
One of these disks is to be a boot disk.
It must contain a kernel that supports IDE drives
and networking ({\ltt{}net600.i}).
The other should be a root disk ({\ltt{}color.gz}).

You can find the disks under the {\ltt{}~djv}
directory (visible from the machine you administer).

{\it Report:} the exact commands you used to build the disks.

Test: Making sure that all other users are off the machine,
place your boot disk in the A: drive and
reboot the machine using the shutdown command.
Follow the instructions, paying careful attention for when to swap disks.
Once you have booted from floppy, examine the {\ltt{}/bin} directory
(see next report section).
Remove your disks from the drive and reboot using the shutdown command.

{\it Report:} the number of entries you found in {\ltt{}/bin}.

\bye
