\input macros
\rightskip=0pt plus 1fill
\input cstuff
\headline{{\bf CECS 476\hfill Homework 6 \hfill Fall 2003}}
\footline{Dennis Volper \hfill 22 September 2003 (Week 4 Lecture 1)\hfill 
Due: 24 September 2003 (Week 4 Lab 2)}
\parindent 0pt

This part of the assignment is designed to familiarize you with the
commands involving Unix processes.

Submission: You will submit a written document.

For Cheetah: 1) how long has it been since it has
been rebooted, 2) how many users are logged in, 3) how busy is the machine
(short, medium and long term load).

For Cheetah: 3) What percentage of the cpu is the operating system
spending in each of user mode, system mode and idle?
4) How much memory is in use and
how much is idle?
5) What 3 processes have consumed the most total CPU since
they were started?

For the machine you administer answer the following:
6) How many processes called {\ltt{}agetty} are running?
7) There are several commands running that include the letters {\ltt{}rpc},
list all those commands.
8) What is the resident set size and total size of the lpd process.

One of the users on your system has created a run-away process.
Use your system administrator privilege to
get rid of the process the process.
(-10 points if you reboot the machine)

Submit: 9) The exact commands you used to examine the processes;
who owned the process, what was it called and what was its
process id number;
and the exact command you used to get rid of the run-away process.

Note: If your machine was rebooted recently, it will have wiped
out the run-away process, and you will have to let me know so
I can start one.

This part of the assignment is designed to familiarize you with the
commands involving virtual memory and swapping.

10) As it is set up,
does your machine use a swap partition or a swap file? 
Report which it is.
Report the full name of the swap partition or file.

11) Report the following in kilobytes:

a) the amount of memory available (minus the kernel),
\break
b) the amount of memory in use,
\break
c) the amount of memory used for buffers,
\break
d) the amount of swap space available,
\break
e) the amount of swap space in use,

(Any dumps of command output will be given a zero.)

On the lab machine you are assigned to administer:

Set up a 4 MB swap file, call it {\ltt{}/swapfile}.
Turn on swapping for that file.

12) Report the exact commands you used to do this.

13) Report the amount of swap space available and the amount in use.

Turn off swapping and remove your file.

Set up your two swap partitions.
Modify your {\ltt{}fstab} so that swapping is turned on automatically
at boot for those partitions.
(Use the {\ltt{}swapon -a} command to make sure you've got everything
setup correctly, else the next step will lock up!!!)
Now that you've tested to make sure things work, reboot
(This is to make sure the ``automatically at boot" is working.)

(If you did the {\ltt{}fdisk} as directed in the previous home work,
your swap partions should be {\ltt{}/dev/hda2} and {\ltt{}/dev/hdc1}.)

14) Report the amount of swap space available and the amount in use.

\bye
