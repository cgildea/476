\input macros
\rightskip=0pt plus 1fill
\input cstuff
\headline{{\bf CECS 476\hfill Homework 12 \hfill Fall 2003}}
\footline{Dennis Volper \hfill 13 October 2003 (Week 7 Lecture 1)\hfill 
Due: 15 October 2003 (Week 7 Lab 2)}
\parindent 0pt

1) This project is about configuring a machine to be on the network.

Before starting this project, ask the instructor to disable networking
on your machine. He will have to reboot to do this.

Get you machine up and onto the network.
Do this by typing commands in one at a time, not by running a script.
You will net enable all of networking, just enough to be
able to ping {\ltt{}jaguar} and {\ltt{}aardvark}.

The hostname, internet address, gateway and netmask of your machine are
on the label on the front.
The cable number for your machine is it's address with the host
part set to all zeros.
The broadcast address for your machine is it's address with the host
part set to all ones.
You should test to ensure your machine is connected to the network.

Report: the exact sequence of commands you used to connect your machine
to the network.

Clean up: to re-enable default networking you must change the mode
of the network initialization files used by {\ltt{}init}.
Type the following command:
\hfill\break
{\ltt{}chmod a+x /etc/rc.d/rc.inet*}
\hfill\break
Now when you reboot {\ltt{}init} will bring the network up automatically.

2) Resolver/DNS. The command {\ltt{}nslookup verity} fails because the full name
of the machine is {\ltt{}verity.ics.uci.edu}.
Fix your machine so that {\ltt{}nslookup verity} works (as well as using
the ``first-name" of all other machines ending in {\ltt{}ics.uci.edu}.

3) Using the {\ltt{}inetd.conf}, disable telnets into your machine.

Report: What did you do to the file.

Cleanup: reenable telnets.

4) Using the tcp wrappers, deny telnets from {\ltt{}panther}. (Be sure to test
that they are still allowed from other machines.)

Report: Which file did you use and what did you put into it.

Cleanup: reenable telnets.

5) Report: List the names of the remote programs available on {\ltt{}jaguar}.
(List each name just once).

6) Allow csc476xx (your account) on {\ltt{}panther} to rlogin as bob 
on the machine you administer without a password.

Report: Which file did you use, on which machine, and what did you put into it.

7) Allow any user from {\ltt{}cougar} to rlogin to the machine you administer 
without a password.
That is, if they are joe on {\ltt{}cougar}, they are allowed to rlogin as
joe on your
machine without a password. Of course joe must be  a user that exists.
Check this out by using your csc476xx account (because it is the only
account you have access to that exists on both machines).

Report: Which file did you use, on which machine, and what did you put into it.
\bye
