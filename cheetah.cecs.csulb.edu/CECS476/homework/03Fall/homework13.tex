\input macros
\rightskip=0pt plus 1fill
\input cstuff
\headline{{\bf CECS 476\hfill Homework 13 \hfill Fall 2003}}
\footline{Dennis Volper \hfill 15 October 2003 (Week 7 Lecture 2)\hfill 
Due: 20 October 2003 (Week 8 Lab 1)}
\parindent 0pt

Since NIS is running, this project mostly involves only
information gathering and configuration examination:

1) What is the NIS domain name of your machine.
Report the command you ran to find this out and the domain name.

2) What NIS server is your machine using.
Report the command you ran to find this out and the server name.

3) How many password lines does NIS deliver on your machine.
Report: the exact command you used and the number of lines.

4) Examine the NIS password information arriving at your machine.
For your group select one of your 476 accounts {\it report} the exact line of 
information that arrives.
You are NOT allowed to use the same yp command you used in the previous
question, use another command to match the line in the password file.
Report the exact command you used to get that information.

5) Examine the {\ltt{}/var/yp} directory tree on your machine.
Report: what in this subtree is dependent upon your domain name.

6) Examine your {\ltt{}yp.conf} file, what is there.

7) Examine the start-up code for the yp client program.
What is the exact (full) pathname of the program that is run
(that means the name starting with the {\ltt{}/}).
Under what condition is this program started?

8) Examine your {\ltt{}nsswitch.conf}.
Is it set up to use yp for passwords and groups?
How do you know?

9)  On cheetah, examine the {\ltt{}/etc/netgroup} file.
Report: How many machines are in the {\ltt{}cslabd} netgroup.

10) On the machine you administer, the groups that {\ltt{}cheetah} is
delivering using NIS are not being used. (This gives you
a hint about one of the previous questions.)
Fix this.
Make the appropriate changes, HUP or restart the {\ltt{}ypbind}.
Report the changes you made.
To assist you in testing this, after the changes have been made, if you
list sue's directories the files should show the group as {\ltt{}fall2003}.
\bye
