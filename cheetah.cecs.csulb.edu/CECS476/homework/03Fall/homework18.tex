\input macros
\rightskip=0pt plus 1fill
\input cstuff
\headline{{\bf CECS 476\hfill Homework 18 \hfill Fall 2003}}
\footline{Dennis Volper \hfill 5 November 2003 (Week 10 Lecture 2)\hfill 
Due: 10 November 2003 (Week 11, Lab 1)}
\parindent 0pt

This project is about ftp.

Because the ftp that comes with the normal distribution is broke,
it doesn't work with yp; we need to install a working version of
ftp. This version is wuftp.

The install must be done as root on your machine.
Change directory to ``{\ltt{}/}"; all tgz's from the install
are installed with this as the base directory.
Extract the compressed tar file {\ltt{}~djv/wuftpd.tgz}.
Run the ``install" script, the command ``{\ltt{}source install/doinst.sh}
will do this.
Make the change to {\ltt{}inetd.conf} descriped in the installing Linux
homework.

1) Log into jaguar.
Ftp to your local machine.
Use the username anonymous.
Get the dummy test file.

Report: On login. It says "Welcome,...";
report the next two words after welcome.

2) Compare the {\ltt{}ls} command that everyone else uses with the
one for anonymous ftp.
Examine the size of the two ls command
and use {\ltt{}ldd} to look at the dynamically loaded libraries.

Report: the difference between the two commands (sizes and dynamic libraries).

3) The current setup of ftp is a little less secure than it
could be.
Some change would make it more secure.
Make these changes.
Then check your changes by logging log into jaguar a second time
and using anonymous ftp to get the dummy test file again.

Report: What changes you made.
If a change was made in a file, report the name of the file and
the result of your changed.
If you changed ownership or protection, report the new owner or mode.
\bye
