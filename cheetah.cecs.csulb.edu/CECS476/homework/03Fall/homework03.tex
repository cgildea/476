\input macros
\rightskip=0pt plus 1fill
\input cstuff
\headline{{\bf CECS 476\hfill Homework 3 \hfill Fall  2003}}
\footline{Dennis Volper \hfill 10 September 2003 (Week2 Lecture 2)\hfill 
Due: 15 September 2003 (Week 3 Lecture 1)}
\parindent 0pt

Purpose: This assignment is to familiarize you with setting up users and logins.

System Admin: When you submit assignment 2, you should ask for (and get)
a system administrators account.

Submission: You will submit a written document.
Put you name, account, machine and system administrators number on your 
document.
The document will contain the exact version of each command you did.
If you did an edit, tell me exactly what change you made to the file. 
You will leave the user login setup on your computer.

On the lab machine, you are assigned to administer set up the
following two accounts.
Do not use a script, do all the steps individually, by hand.

{\bf Account 1}:

Name {\ltt{}bob}
\break
Password: access31
\break
UID: {\ltt{}13101}
\break
Default group: {\ltt{}cecsu}
\break
Home directory: {\ltt{}/home/bob}

{\bf Account 2}:

Name {\ltt{}sue}
\break
Password: access32
\break
UID: {\ltt{}13102}
\break
Default group: {\ltt{}cecsg}
\break
Home directory: {\ltt{}/home/sue}

For {\it both} accounts:

Shell: {\ltt{}/bin/csh}
\break
Home directory contents, 3 files:
{\ltt{}.cshrc}, {\ltt{}.login} and {\ltt{}demo}.
\break
The {\ltt{}.cshrc} and {\ltt{}.login} files should be identical to those in
the home directory of the {\ltt{}djv} account.
You should create the {\ltt{}demo} file using an editor
and it should contain one line of text saying,
``Hi there, welcome to CECS". 
(Note, this last file has no particular purpose other than
to make sure you know how to set up permissions correctly.)

The file ownership and group access of all files and directories
should be correctly set.

Testing:
\break
When the instructor grades your homework the following tests will be
performed on your assignment:
Login as your new user. Run {\ltt{}pwd} to make sure the home directory
is where it is supposed to be.
Run. {\ltt{}ls -al} to make sure the files are there, that they have
the correct user and group and that your user doesn't own files
outside the user's home directory.
\bye
