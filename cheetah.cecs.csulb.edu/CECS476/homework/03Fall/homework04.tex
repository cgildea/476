\input macros
\rightskip=0pt plus 1fill
\input cstuff
\headline{{\bf CECS 476\hfill Homework 4\hfill Fall 2003}}
\footline{Dennis Volper \hfill 15 September 2003 (Week 3 Lecture 1)\hfill 
Due: 17 September 2003 (Week 3 Lab 2)}
\parindent 0pt

On all assignments, be sure to indicate the name of the
machine you are assigned to administer.

This first question of this assignment emphasizes shell programming,
the remaining questions have you examine the file system.

Make sure the {\ltt{}bob} exists on your 
machine.

As bob, write a shell script that examines all the processes on the
system and ``reports" c-shell processes
into a file called {\ltt{}cshlog}.
First, this script should append to the {\ltt{}cshlog} the {\ltt{}date}.
Then, it should do a {\ltt{}ps aux}, and record all lines (use {\ltt{}grep})
containing the letters csh should be appended to the {\ltt{}cshlog}.
Your script file should be called {\ltt{}cpeek}

Test your shell script by
running your command several times with various users logged 
in (for example root and your csv account)
(i.e., just make sure it works).

1) Report: the exact contents of {\ltt{}cpeek}.
(Do NOT report the contents of your cshlog.)

On the machine you are assigned to administer:
Login using your user account and report the following.

2) What file systems are mounted on your machine and where (mount)?

3) On the root ({\ltt{}/}) file system, how much disk is available (df)?

4) In your home directory, how much space have you used (not much yet) (du)?

5) What partition does your machine use as a swap partition?
What file from the {\ltt{}/etc} directory determines this
and report the exact contents of the line that determines the swap space.

Using the system administrator account,
examine the file system of the linux partition on your hard drive
({\ltt{}/dev/hda1}) using the {\ltt{}dumpe2fs} command.

6) Report: The file system state, the block size and the number of groups
in that file system.

7) {\bf fdisk}:
For the disk you administer describe the partitions that are on the disk?
For each partition {\it report}, where it starts, where it ends, and what kind
of a partition it is (Linux, DOS, swap).
Be very careful with this one, you have to be root to run fdisk,
DO NOT MODIFY or WRITE anything with fdisk; 
you could erase your hard drive.

\bye
