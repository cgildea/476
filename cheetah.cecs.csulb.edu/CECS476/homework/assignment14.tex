\input macros
\rightskip=0pt plus 1fill
\input cstuff
\headline{{\bf CECS 476\hfill Assignment 14 \hfill Spring 2014}}
\footline{Nathan Pickrell \hfill 11 March 2014 (Week 8 Lecture 1)\hfill 
Due: 13 March 2014 (Week 8 Lab 2)}
\parindent 0pt

This project is about NFS (Network File System).

Examination items:

1) What does {\ltt{}exports} file on {\ltt{}jaguar}.
What directories does it export, to what machines and with what options.
For the ``what machines", give an English description the machines, i.e.
figure out what the netmasks mean.
WARNING: you need to know the options to get the action items to work.

2) The machine you administer mounts two directories from cheetah.
{\bf Report:} the fstab entries on your machine that cause these mounts.

3) Cheetah is running the automounter.
Look at cheetah. 
{\bf Report:} what NFS directories does cheetah actually have mounted 
({\ltt{}df} or {\ltt{}mount}).
(Note as our discussion of the automounter pointed out, the answer will
depending on which of these directories has been accessed recently.)
Now do an "{\ltt{}ls ~volper}" and report any change in the directories
cheetah has mounted. (Again, there will be no change if some
one did this just before you did.)
Examine the automount configuration ({\ltt{}/etc/auto.master}), it
uses the indirect format.
This means it will refer to some additional files which you will need to
examine as well.
3a) {\bf Report:} what options does cheetah use when it tries to mount the
directory containing {\ltt{}~volper} and what is the name of the file
that contained those options.
3b) {\bf Report:} what options does cheetah use when it tries to mount the
directory from your machine and what is the name of the file
that contained those options.
WARNING: options not listed default.
You should examine the {\ltt{}mount}
manual entry and know what the defaults are;
this is critical to getting the following questions to work.

Action items:

4) Configure your fstab so that it attaches the directory
{\ltt{}/sdb/slack13.37-64/slackware64} from jaguar on your local directory tree
as {\ltt{}/mnt} whenever it boots.
Jaguar is already configured to make this directory available to your
machine, you do not need to modify jaguar.
Your {\ltt{}/mnt} directory should already exist.

{\bf Report:} the file you changed on your local machine and the line
you added to it.

5) Configure your machine so that it makes the directory
{\ltt{}/etc} available to cheetah.
Cheetah will access this directory as {\ltt{}/net/lab/labxx} where {\ltt{}labxx}
is the name the machine you are assigned to administer.
Cheetah is already configured to mount your {\ltt{}etc} directory,
you do not need to make any modifications to cheetah.
Because your {\ltt{}exports} was empty when you last booted;
you will also need to start {\ltt{}rpc.nfsd} and {\ltt{}rpc.mountd}.
(I recommend using the {\ltt{}rc.nfsd} script to do this.)

Testing: Log into cheetah and run {\ltt{}ls /net/lab/labxx}.
Because cheetah is running the automounter
this {\ltt{}ls} will cause your directory to be mounted.
Now run {\ltt{}df}, you should see the {\ltt{}/etc} from your machine mounted.

{\bf Report:} the files you changed on your local machine and give the lines 
you added to them.
Also report the part of the disk free report from cheetah that applies to 
the {\ltt{}/etc} from your machine.

6) Configure the automounter on your machine so that it will mount the
{\ltt{}/load} directory from {\ltt{}jaguar} on {\ltt{}/opt} (that directory
should already exist). Specifically {\ltt{}ls /opt/load} should display
the directory from {\ltt{}jaguar}. Use an indirect map.
Automount is by default disabled so you will need to enable it by marking
the start script executable ({\ltt{}chmod a+x /etc/rc.d/autofs}).
You will need to use that script to start and reload the automounter.
(Reload when you make changes to the files.)

{\bf Report:} The line you added to {\ltt{}auto.master} and the entire
indirect file you created.
\bye
