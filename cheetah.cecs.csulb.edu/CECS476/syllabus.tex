\input macros.tex
\footline{{\rm Nathan Pickrell --  January 2014 \hfill}}
\centerline{{\bf SYLLABUS --- CECS 476}}
\vskip 5pt
{\obeylines\parindent 0pt
{\bf System and Network Administration}\hfill Spring 2014
lass: ECS 405 (TTh 5:00--5:50PM) \hfill Course code: 13289 \hfill Lab: ECS 405 (TT 6:00--7:15PM)

Instructor: Nathan Pickrell \hfill Lab Hours: ECS 405  TTh 5:00-8:15PM (or on request if earlier)

}
 
\vskip 5pt
\centerline{\bf Course Objectives}
 
To introduce the principles behind a multiuser operating system with network
access.
To teach the of administration of computers with such operating systems.
The course will cover both how the Unix system works
and the tasks the system administrator must perform.

\vskip 5pt
\centerline{\bf Prerequisites }

A knowledge of Unix and operating systems (e.g., CECS 326).
The ability to read technical material (manuals, documentation).
Some experience in computing (several upper division courses).
The ability to follow instructions.
 
\vskip 5pt
\centerline{\bf Books }
 
A book is helpful if you have trouble understanding the manual entries
and the lecture.
The Nemeth book is listed as the official text, but there are several
good books in different areas of system administration.

{\it Unix System Administrators Handbook}
by Nemeth et.al.
This book covers several versions of Unix, including Linux.
It covers many topics, but sometimes doesn't cover them
as completely as you would like.

{\it Running Linux}
by Welsh et.al.
This book contains information specific to Linux.
On some topics it has more information than Nemeth,
but it does not cover as many topics.

In addition you may find books from the following list helpful. 
Most can be found cheaply at major
computer stores or ordered on-line.

% {\it Linux System Administration}
% {\it{}Linux Configuration and Installation}
% {\it The Linux Internet Server}, 
% {\it{}Linux Network}.
% This is a series of books from IDG.
% There is overlap, but they cover the topic completely.

{\it Linux System Administration Handbook}
by Komarinski and Collett. 
Good in the networking area, 
but a little shallow in non-networking places.

% {\it Unix System Administrator's Edition} from SAMS publishing.
% Good, but covers all versions of Unix so the Linux information
% is scattered throughout the book.

{\it Essential System Administration} by Aeleen Frish (O'Reilly).
Good, but aimed at non-Linux versions of Unix.

{\it Linux Network Administrator's Guide} by Olaf Kirch (O'Reilly).
Very good, but only covers the networking part of Linux.

\vskip 5pt
\centerline{\bf Other Materials}

You will be issued accounts for the system administration lab.
This account is separate from any other department or university
accounts you are issued and is good for the semester only.
Online manuals will be available using those accounts.
How-To files are available for many functions.
Additional materials will be available online.
% the first set by being handed out, later sets through the copy center 
% of the campus book store.
Copies of the lecture slides are available in the book store.
Source for the lecture slides is available online in TeX format
from the computer {\tt{}cheetah} under the directory
{\ltt{}~volper/classes/476}.
That directory also includes lots of other useful items including copies
of the old exams in {\ltt{}~volper/classes/476/old/exams}.

\vskip 5pt
\centerline{\bf Structure of the Course}
 
There will be a large number (approximately one per lab) of 
assignments, one midterm, and a final.
The assignments will be worth 40\% of the grade,
each exam will be worth 30\% of the grade.
I recommend that you maintain a notebook documenting those actions
you have taken as a system administrator.
The exams will be open note, so the better organized your notebook
the more helpful you will find it during the exams.

\vfill\eject
\centerline{\bf Laboratory}

Your working environment for System Administration is part of the CSULB network.
There are many machines on this network, they are accessible in
many different ways.
The lab will be ECS 405. It contains the machines reserved for this course.
These machines are accessible over the network,
but in a restricted fashion so as to prevent the machines from being hacked.
You may login to these machines from any cecs computer,
but you will need to login to {\tt cheetah}
first, then telnet/ssh to your Linux machine.
If you are off campus you may access {\tt cheetah} only through ssh.
We suggest using putty if you are working from a Windows computer.
We recommend you perform your actual system administration in the laboratory
during lab hours.
You will be issued system administration privileges;
exercise these privileges only for the purposes described in the assignments.

\vskip 5pt
\centerline{\bf Assignments}

Work is due in the lab on the date indicated on the assignment.
If the assignment indicates there is a demonstration required
that demonstration must performed at the time the assignment is submitted.
Assignments may be submitted early (no penalty).
Late submissions will be penalized 5 points per lab that they are late.
To allow for illness and emergencies,
the first 50 points of late penalty you
accrue during the semester will be waived.
If you are examining your grade, the number of waiver points you have
remaining is under the column labeled ``{\ltt{}wv}".
No work will be accepted after the last lab of the semester.

In general, you should leave the results of your assignment running or
active on your machine so the instructor can verify your work.
In a few cases you will be given specific ``clean up" instructions
to shutdown particular parts of assignments which have the potential
to clog the system.

You are expected to run your tests and make sure your assignments work
before submission.
One re-submission will be allowed during the semester, you will receive
the average of the scores of the two submissions. The amount of late
penalty will be based on the date of the second submission.
This is intentionally severe, I don't encourage re-submissions, test your
work before submitting the first time.

To view your scores, login to {\ltt{}cheetah} and run {\ltt{}getscore}.

\vskip 5pt
\centerline{\bf Course Summary}

In general, we will cover a set of principles used by the Unix operating system,
you will look at the operating system to observe details on these principles,
then you will perform system administration tasks based on these principles.
Much of the material for your work will come from the on-line documentation.

\vskip 5pt
Final Exam: 
5:00--7:00, Thursday, 15 May 2014

\def\week#1{\par\hangindent 0.7in {\indent\llap{\bf #1 \enspace}
\ignorespaces}}

\bye
